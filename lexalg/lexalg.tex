\documentclass[../paper.tex]{subfiles}

\begin{document}
\section{The Main Algorithm}

\begin{enumerate}[label*=\arabic*.]
  
  \item Start with $P = \{1\}$

  \item If $|P| < d$, append the next odd prime to the end of
    $P$. \label{loop}

  \begin{enumerate}[label*=\arabic*.]

    \item If $b_{1}(P) \geq 2$, then truncate the last prime
      from $P$. Let $p$ be the minimum prime such that $b_{1}(P +
      p) < 2$. Append $p$ to the end of $P$. 

    \item Go to step {\ref{loop}}

  \end{enumerate}

  \item If $b_{\infty} \leq 2$, then follow the 'Failure
        Procedure' with $P$. Afterwards, if $P = \emptyset$,
        terminate the algorithm. Otherwise, go to step
        {\ref{loop}}.

\end{enumerate}

(At this point, we know the following facts: $|P| = d$ and
$b_{\infty}(P) > 2$. We are now in a position to use the exponent
algorithm to find any OPN's whose prime divisors compose $P$.)

\begin{enumerate}[resume]  

  \item Find all OPN's whose divisors compose $P$, using the
        Exponent Algorithm. Record these integers.

  \item If at least one OPN was found in the previous step,
        replace the last prime of $P$ with its prime successor.
        Go to step {\ref{loop}}. 

  \item Otherwise, follow the 'Reverse Procedure*' on $P$. Denote
        the prime that was last truncated from $P$ as $q$. 
        
  \item If $|P| \geq d - 1$, then follow the 'Failure Procedure'
        on $P$ and go to step {\ref{loop}}.

  \item If $q \leq Cap_d(P)$, then append the next prime after $q$
        to $P$. Otherwise, follow the 'Failure Procedure' for $P$.

  \item if $P = \emptyset$, then terminate. Otherwise, go to step
        {\ref{loop}}
\end{enumerate}

THE FAILURE PROCEDURE:

\begin{enumerate}[label*=\arabic*.]

  \item Let $P' := P$. This is so we do not modify the original
        $P$ during the procedure, as we will reference $P$.  

  \item Remove the last prime from $P'$.

  \item \label{fail_loop}
        Consider all $Q$-composed OPN's with $d$ prime divisors such
        that $Q$ is a lexicographical predecessor of $P$. Denote
        this set of OPN'S as $A$.(Since the algorithm works in
        lexicographical order, the algorithm has determined $A$
        for us) 

  \item if there does not exists an $a \in A$ such that $a$ is
    $P'$-initiated, go to step 2.

  \item Set $T(P') := True$. This value is false by default.

  \item Truncate elements from $P'$ until $T(P')$ is false.
        Clearly at least one element must be truncated.
        Afterwards, set $P := P'$. Terminate the algorithm.
\end{enumerate}

THE BACKUP PROCEDURE: Similar to the failure procedure but with
less steps and deletions.

\begin{enumerate}[label*=\arabic*.]

  \item Let $P' := P$. This is so we do not modify the original
        $P$ during the procedure, as we will reference $P$.  

  \item Remove the last prime from $P'$.

  \item \label{fail_loop}
        Consider all $Q$-composed OPN's with $d$ prime divisors such
        that $Q$ is a lexicographical predecessor of $P$. Denote
        this set of OPN'S as $A$.(Since the algorithm works in
        lexicographical order, the algorithm has determined $A$
        for us)

  \begin{enumerate}[label*=\arabic*.]

    \item if there exists an $a \in A$ such that $a$ is
    $P'$-initiated, then set $P := P'$. Now, terminate.

    \item If there does not exist such an $a$, Go to step 2 %{\ref{fail_loop}}

  \end{enumerate}

\end{enumerate}


%Todo : Start writing up algorithm in notebook.
\begin{proof}
Will use Continuity of Primitive Abundant Numbers and Prine
Divisibility of Abundant Numbers.

We will also do a proof by induction, on the sequences of primes.
We will lexicographically order the sequences of primes to do
this.

Suppose that $P$ and $Q$ are sequences of $d$ primes such that $P$
and $Q$ compose the prime factorization of two odd primitive
abundant numbers. Suppose that $Q$ is the greatest predecessor of
$P$ with the primitive abundant property. We want to show that if
the algorithm collects $P$, then it must collect $Q$. 

For the base case: We need to use Theorem {\ref{Primechops}} to
construct the smallest sequence and show that the algorithm
collects it. Would be nice to show that the smallest sequence
starts with a $3$

To prove this we have two cases: $P$ and $Q$ share all but the
last prime factor or $P$ and $Q$ have less elements in common
prime factor.

In the case where $P$ and $Q$ share all but the prime factor, we
will argue the fact that $P$ is the immedient predecessor of $Q$,
i.e. the last element of $Q$ is the next prime after the last
element of $P$. This is beacause of Theorem {\ref{Continuity}}. In
this case, $Q$ will definately be the next prime sequence checked
and therefore the algorithm will discover $Q$.

In the case where $P$ and $Q$ have at least two elements not in
common, we want to use Theorem {\ref{Divisibility}} and show that
the divisors of $P$ are fixed at a certain size.

\end{proof}

\end{document}
