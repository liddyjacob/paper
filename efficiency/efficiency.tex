\documentclass[../paper.tex]{subfiles}

\begin{document}
\section{Increasing Computational Efficiency}

The exponent algorithm is a very computationally expensive
algorithm, and we want to be able to reduce the amount in which it
is used. 

Let $P$ be an increasing sequence of primes. We define the set
$\mathcal{E}_P$ as 

$$\mathcal{E}_P = \{E | \nu(P, E) \text{ is an OPN}\}$$.

We make the follwoing claim:

\begin{theorem}[First Efficiency Theorem] 
Let $P$ be an increasing sequence of odd primes. Suppose
that $p$ is a prime greater than the last prime of $P$ and $q$ a
prime greater than $p$. If $\mathcal{E}_{(P + p)} \neq 
\emptyset$ and $\mathcal{E}_{(P + p)}$ is a subset of 
$\mathcal{E}_{(P + q)}$, then $\mathcal{E}_{(P + p)} = 
\mathcal{E}_{(P + q)}$.
\end{theorem}

The above theorem seems to apply to a massive amount of cases,
especially as the last prime grows larger and larger. Because of
this, this Theorem allows us to avoid the exponent algorithm,
which is a very computationally expensive algorithm. This theorem
allows all primitive abundant numbers with 5 divisors to be
computed in just a couple of minutes, where previously it took
about an hour. Using multithreading abilities on certain
processors, the time can be cut even further.


\begin{proof}
  Suppose, to the contrary, that the theorem is not true. Then
there exists some $E' \notin \mathcal{E}_{P + p}$ such that $\nu
(P + q, E')$ is an OPN. As $E'$ is not in $\mathcal{E}_{P + p}$,
we have that $\nu (P + p, E')$ is not an OPN. However, $\nu (P +
p, E')$ is abundant as $p < q$. Thus $\nu (P + p, E')$ has a
divisor $a$ such that $a$ is an OPN. 

  We will now show that the prime divisors of $a$ compose $P + p$.
We know that $p$ must divide $a$, for if $p$ does not divide $a$,
then $a$ divides $\nu (P + q, E')$. Let $k$ be the maximum
integer such that $p^k$ divides $a$. We know that $a / p^k$ is
deficient as it is a divisor of $\nu (P + q, E')$. It is not hard
to show that all primes of the set $P$ must divide $a$. Thus the
prime factors of $a$ compose $P + p$. 

  Let $a = \nu(P + p, E'')$. We can find some sequence of positive
integers $E''$ such that $a = \nu (P + p, E'')$ as the prime divisors
of $a$ compose $P + p$. Thus $E'' \in \mathcal{E}_{P + p}$. By
hypotheses, we have that $\nu(P + q, E'')$ is an OPN. This,
however, is a contradiction with the fact that $\nu(P + q, E'')$
is a proper divisor of $\nu(P + q, E')$.
\end{proof}


\begin{theorem}[Second Efficiency Theorem] 
Let $P$ be an increasing sequence of odd primes. Suppose
that $p$ is a prime greater than the last prime of $P$ and $q$ a
prime greater than $p$. If $\mathcal{E}_{(P + p)} \neq 
\emptyset$ and $\mathcal{E}_{(P + p)}$ is a subset of 
$\mathcal{E}_{(P + q)}$, then $\mathcal{E}_{(P + p)} = 
\mathcal{E}_{(P + q)}$. In fact, for any prime $r$ such that $p <
r < q$, we have that $\mathcal{E}_{(P + p)} = \mathcal{E}_{(P +
r)} = \mathcal{E}_{(P + q)}$.
\end{theorem}

There are be times where we can
calculate a few thousand primes(or more) ahead and skip many prime
numbers at a time. It would be easy to create a smart algorithm to
skip the appropriate amount of primes, an algorithm that will
increase the number of skips on each success, and decrease on the
correct number on each failure. 

Note: This theorem takes what runs in possibly a year and allows
the algorithm to run in just a few hours.

\begin{proof}

Suppose that $\mathcal{E}_{(P + p)} = \mathcal{E}_{(P + q)}$.
Suppose, to the contrary, that $\mathcal{E}_{(P+r)} \neq
\mathcal{E}_{(P + p)}$. Then there exists some $E \in
\mathcal{E}_{(P + r)} | E \notin \mathcal{E}_{(P + p)}$, or there
exists some $E \notin \mathcal{E}_{(P + r)} | E \in \mathcal{E}_{(P +
p)}$. 

Suppose the former case is true. If $m = \nu(P + q, E)$ is abundant,
then $m$ must not be primitive. Note that any divisor of $\nu(P +
r, E)$ is deficient and can be written in the form $\nu(P + r, E')
| e_i \in E, e'_i \in E' => e_i \geq e'_i$. We want to consider
an $E'$ with nonzero elements only(separate case when this does
not happen). Such an $E$ 

See algorithms notebook: The outline of the proof lies there.

Thus a divisor of $m$ is
primitive. => ... => m is primitive abundant for...
Clearly, $\nu(P + p, E)$ is
abundant but $\nu(P + p, E)$ must not be deficient. 

\end{proof}

Through implimenting this algorithm, we were able to find the
number of OPN's with 6 divisors. That number is: (TO BE COMPUTED)


The largest odd primitive abundant number with 6 divisors is:
$3^38 5^28 17^16 257^8 65537^4 4294967291^2$. It has 116 decimal
digits.



\end{document}
