\documentclass[../paper.tex]{subfiles}

\begin{document}
\section{Increasing Computational Efficiency}

The exponent algorithm is a very computationally expensive
algorithm. If we can avoid executing the exponent algorithm, then
the amount of time it takes to find all OPAN's with d divisors will
be drastically reduced. 
We make the following claim:

\begin{lem}[Efficiency Lemma]\label{first}
Let $P$ be an increasing sequence of odd primes. Suppose
that $p$ is a prime greater than the last prime of $P$ and $q$ a
prime greater than $p$. If $\mathcal{E}_{(P + p)} \neq 
\emptyset$ and $\mathcal{E}_{(P + p)}$ is a subset of 
$\mathcal{E}_{(P + q)}$, then $\mathcal{E}_{(P + p)} = 
\mathcal{E}_{(P + q)}$.
\end{lem}

\begin{proof}
  Suppose, to the contrary, that the lemma is not true. Then
there exists some $E \notin \mathcal{E}_{P + p}$ such that $\nu
(P + q, E)$ is an OPAN. As $E$ is not in $\mathcal{E}_{P + p}$,
we have that $\nu (P + p, E)$ is not an OPAN. However, $\nu (P +
p, E)$ is abundant as $p < q$. Hence there exists a divisor of  
$\nu (P + p, E)$ which is an OPAN. Suppose $a$ is such a divisor.  

  We will now show that the prime divisors of $a$ compose $P + p$.
We know that $p$ must divide $a$, for if $p$ does not divide $a$,
  then $a$ is an abundant proper divisor of $\nu (P + q, E)$(a contradiction). 
Let $k$ be the maximum
integer such that $p^k$ divides $a$. We know that $a / p^k$ is
deficient as it is a divisor of $\nu (P + q, E)$. Using a similar argument,
we can show that all primes of the set $P$ must divide $a$. 
Thus the prime factors of $a$ compose $P + p$. 

  We have shown that each prime number in $P + p$ divides $a$, and
therefore $a$ can be written in the form $a = \nu(P + p, E')$ for
some sequence of positive integers $E'$. As $a$ is an OPAN, we
have that $E' \in \mathcal{E}_{(P +p)}$. By hypotheses,
$\mathcal{E}_{(P + p)}$ is a subset of $\mathcal{E}_{(P + q)}$.
This implies that $E' \in \mathcal{E}_{(P + q)}$. Therefore $a_q = 
\nu(P + q, E')$ is abundant. However, $a_q$ is a divisor of
$\nu (P + q, E)$, and therefore $a_q$ is deficient, a
contradiction.

\end{proof}


\begin{theorem}[Efficiency Theorem] \label{eff}
Let $P$ be an increasing sequence of odd primes. Suppose
that $p$ is a prime greater than the last prime of $P$ and $q$ a
prime greater than $p$. If $\mathcal{E}_{(P + p)} \neq 
\emptyset$ and $\mathcal{E}_{(P + p)}$ is a subset of 
$\mathcal{E}_{(P + q)}$, then for any prime $r$ such that $p <
r < q$, we have that $\mathcal{E}_{(P + p)} = \mathcal{E}_{(P +
r)} = \mathcal{E}_{(P + q)}$.
\end{theorem}

Theorem {\ref{eff}} seems to apply to a massive amount of cases,
especially as the last prime grows larger and larger. Because of
this, Theorem {\ref{eff}} allows us to avoid the exponent algorithm,
which is a very computationally expensive algorithm. This theorem
allows all primitive abundant numbers with 5 divisors to be
computed in just a couple of minutes, where previously it took
about an hour. Using multithreading abilities on modern
processors, the time can be cut even further.

\begin{proof}
If $\mathcal{E}_{(P + p)} = \{\{1,1,1, ..., 1\}\}$, then the
theorem is trivial. Henceforth we shall assume that for each 
element E in $\mathcal{E}_{(P + p)}$, E contains at least one term
greater than or equal to 2.

Suppose that $\mathcal{E}_{(P + p)} = \mathcal{E}_{(P + q)}$.
Suppose, to the contrary, that $\mathcal{E}_{(P+r)} \neq
\mathcal{E}_{(P + p)}$. Then there exists some $E \in
\mathcal{E}_{(P + r)}$ such that $E \notin \mathcal{E}_{(P + p)}$, or there
exists some $E \notin \mathcal{E}_{(P + r)}$ such that 
$E \in \mathcal{E}_{(P + p)}$. 

Suppose the former case is true. Let $E$ be an element of
$\mathcal{E}_{(P + r)}$ such that $E \notin \mathcal{E}_{(P + p)} =
\mathcal{E}_{(P + q)}$. It is easy to show that $\nu(P + q, E)$ is
not abundant. If $\nu(P + q, E)$ were abundant, then $\nu(P + q,
E)$ would be forced to be primitive abundant which would imply 
$E \in \mathcal{E}_{(P + q)}$. It is also easy to show 
$n_p = \nu(P + p, E)$ is abundant. Since $d = \nu (P + p,
\{1,1,1,...,1\})$ is deficient and divides the abundant number $n_p$, 
there exists a primitive abundant odd which is a multiple of $d$
and a divisor of $n_p$. Such a divisor would be in the form $\nu(P
+ p, E')$ for some sequence of positive integers $E'$. This would
imply $E' \in \mathcal{E}_{(P + p)}$. However, this would create a
contradiction as $E' \notin \mathcal{E}_{(P + q)}$ and
$\mathcal{E}_{(P + p)} = \mathcal{E}_{(P + q)}$. 

Now suppose there exists some $E \notin \mathcal{E}_{(P + r)}$ 
such that $E \in \mathcal{E}_{(P + p)}$. From this we know that
$E \in \mathcal{E}_{(P + q)}$ and therefore $n = \nu (P + r, E)$ is
abundant. We also know that for each divisor $\nu (P + r, E')$ of
$n$, we have that $2 > \nu (P + p, E') > \nu(P + r, E')$. Thus $n$
is a primitive abundant number, as each of its divisors are
deficient. This implies that $E \in \mathcal{E}_{(P + r)}$, a contradiction.
\end{proof}

\end{document}
