\documentclass[../paper.tex]{subfiles}

\begin{document}
\section{Increasing Computational Efficiency}

The exponent algorithm is a very computationally expensive
algorithm, and we want to be able to reduce the amount in which it
is used. 

Let $P$ be an increasing sequence of primes. We define the set
$\mathcal{E}_P$ as 

$$\mathcal{E}_P = \{E | \nu(P, E) \text{ is an OPN}\}$$.

We make the follwoing claim:

\begin{theorem}[First Efficiency Theorem] 
Let $P$ be an increasing sequence of odd primes. Suppose
that $p$ is a prime greater than the last prime of $P$ and $q$ a
prime greater than $p$. If $\mathcal{E}_{(P + p)} \neq 
\emptyset$ and $\mathcal{E}_{(P + p)}$ is a subset of 
$\mathcal{E}_{(P + q)}$, then $\mathcal{E}_{(P + p)} = 
\mathcal{E}_{(P + q)}$.
\end{theorem}

The above theorem seems to apply to a massive amount of cases,
especially as the last prime grows larger and larger. Because of
this, this Theorem allows us to avoid the exponent algorithm,
which is a very computationally expensive algorithm. This theorem
allows all primitive abundant numbers with 5 divisors to be
computed in just a couple of minutes, where previously it took
about an hour. Using multithreading abilities on certain
processors, the time can be cut even further.

\begin{proof}
  Suppose, to the contrary, that the theorem is not true. Then
there exists some $E' \notin \mathcal{E}_{P + p}$ such that $\nu
(P + q, E')$ is an OPN. As $E'$ is not in $\mathcal{E}_{P + p}$,
we have that $\nu (P + p, E')$ is not an OPN. However, $\nu (P +
p, E')$ is abundant as $p < q$. Thus $\nu (P + p, E')$ has a
divisor $a$ such that $a$ is an OPN. 

  We will now show that the prime divisors of $a$ compose $P + p$.
We know that $p$ must divide $a$, for if $p$ does not divide $a$,
then $a$ divides $\nu (P + q, E')$. Let $k$ be the maximum
integer such that $p^k$ divides $a$. We know that $a / p^k$ is
deficient as it is a divisor of $\nu (P + q, E')$. It is not hard
to show that all primes of the set $P$ must divide $a$. Thus the
prime factors of $a$ compose $p$. 

  Let $a = \nu(P + p, E'')$. We can find some sequence of positive
integers $E''$ such that $a = (P + p, E'')$ as the prime divisors
of $a$ compose $P + p$. Thus $E'' \in \mathcal{E}_{P + p}$. By
hypotheses, we have that $\nu(P + q, E'')$ is an OPN. This,
however, is a contradiction with the fact that $\nu(P + q, E'')$
is a proper divisor of $\nu(P + q, E')$.
\end{proof}


\end{document}
