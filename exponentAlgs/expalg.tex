%\documentclass[../paper.txt]{article}
\documentclass[../paper.tex]{subfiles}

\begin{document}

\section{Exponent Theorems and Algorithm}

In this section we will understand some of the conditions that
exponents of primitive abundant odd numbers must follow. In order
to do this we first must define some functions and prove theorems 
that will allow us to determine some meaningful information about
exponents of primitive abundant odds.

We define the order of a divisor d (with respect to some multiple
$n$) as the difference in the sum of the exponents between a
divisor and some multiple of that divisor. Suppose $n = p_0^{m_0}
p_1^{m_1} ... p_{d-1}^{m_{d-1}}$ is an integer and $d = p_0^{\ell_0}
p_1^{\ell_1} ... p_{d-1}^{\ell_{d-1}}$. As $d$ is a divisor of
$n$, we have that $m_i \leq \ell_i$ for each $i$. The
\textit{order of $d$ with respect to $n$} is defined as $\sum {m_i
- \ell_i}$. If $d$ is not a divisor of $n$, then the order of $d$
is undefined and meaningless. If the prime factorization of $d$ is
missing some prime factor, then the exponent of that prime factor
is zero.

A related notion is \textit{incomplete primes}. Let $n$ be a
number. Let $d$ be a divisor of that number.
Write $n$ and $d$ as $n = p_0^{m_0} p_1^{m_0}...
p_{d-1}^{m_{d-1}}$ and $d$ as $d = p_0^{m'_0} p_1^{m'_0} ...
p_{d-1}^{m'_{d-1}}$. We say a prime $p_i$ is incomplete in
$d$ if $m'_i < m_i$. When the exponent of a incomplete prime is
increased by one, we get another divisor $d'$ of $n$. In fact, 
this new divisor $d'$ has order $1$ less than $d$.

Yet another notion is \textit{crucial primes}. Let $n = (P,E)$ 
be a number, in its factored form above. Suppose that $p_i \in P$.
If $p_i * n = p_0^{m_0} p_1^{m_1} ... p_{i}^{m_i + 1} ...
p_{d-1}^{m_{d-1}}$ is abundant, then $p_i$ is a crucial prime in
$n$.

We want to define $G_{\Delta_{+}}(P,E)$ as a function which lists
the indicies $i$ of prime/exponent pairs in sequences P and E
in order of least $\Delta_{+}(p_i, e_i)$ to greatest. The tuple
(P, E) is ment to represent some integer $n = p_0^{m_0} p_1^{m_1}
... p_{d-1}^{m_{d-1}}$ with prime factors $P = \{p_0, p_1, ..., 
p_{d-1}\}$ and respective exponents $E = \{m_0, m_1, ...,
m_{d-1}\}$ 
\\

To demonstrate the above function, we calculate 
$G_{\Delta_{+}}(\{3,5,11\}, \{1,2,1\})$.

First we calculate each of $\Delta_{+}(3,1)$, $\Delta_{+}(5,2)$, 
and $\Delta_{+}(11,1)$. We find that 
$\Delta_{+}(3,1) = \frac{13}{12}$,
$\Delta_{+}(5,2) = \frac{156}{155}$, and
$\Delta_{+}(11,1)= \frac{133}{132}$.
Thus the order of $\Delta_{+}$ is as follows:

$$ \Delta_{+}(3,1) > \Delta_{+}(11,1) > \Delta_{+}(5,2)$$

Giving the primes indicies based off its order in the prime
factorization of $3^1 5^2 11^1$ starting at zero, we obtain that

$$G_{\Delta_{+}}(\{3,5,11\}, \{1,2,1\}) = \{1,2,0\}$$
\\

Note that $G_{\Delta_+}$ has the following property:

\begin{theorem}\label{crucial_thm}
Let $G_{\Delta_+}{P,E} = G$ for some integer $n = (P,E)$
Suppose $i$ appears before $j$ in $G$. If $p_i$ is crucial in n,
then $p_j$ is crucial in $n$. In otherwords, there exists some $g$
such that all elements that come before $G[g]$ are not crucial and
all elements that come after $G[g]$ are crucial.
\end{theorem}

The above theorm states that $G_{\Delta_+}(P,E)$ 'looks like'

$$ G_{\Delta_+}(P, E) = (\text{... non-crucial indicies} 
                        | \text{crucial indicies...}) $$

\begin{proof}

Let $n = p_0^{m_0} ... p_{d-1}^{m_{d-1}} = (P, E)$. Suppose that
$p_i$ is crucial in $n$. Note $\Delta_+(p_i, m_i) < \Delta_+(p_j,
m_j)$. Thus,

\begin{equation}
\begin{split}
2 < b(p_i * n) & = b(n)\Delta_+(p_i, m_i) \\
               & \leq b(n)\Delta_+(p_j, m_j) = b(p_j * n)
\end{split}
\end{equation}

Thus $p_j$ is crucial in $n$.

\end{proof}


\begin{theorem}\label{del_minus_thm}
Suppose a number $n = p_1^{m_1} p_2^{m_2} ... p_d^{m_d}$ is 
abundant. Then $n$ is primitive if and only if all of its first
order divisors are deficient.
\end{theorem}

\begin{proof}
	First suppose that $n$ is a primitive abundant. Then any proper 
divisor of $n$ is deficient. As any first order divisor is a
proper divisor, we have that this direction is trivially true.

	Suppose that $n$ is abundant, and that all first order divisors
are deficient. Note that any proper divisor of $n$ must divide
some first order divisor of $n$. Also, any divisor of a deficient 
number is deficient. Since and all proper divisiors of $n$ divide 
a first order divisor, we have that all divisors of $n$ divide a
deficient number. Thus $n$ is a primitive abundant number, proving
the theorem.  
\end{proof}

Suppose $P = \{p_0, p_1, ...,p_{d-1}\}$ are primes and 
$E = \{e_0, e_1, ..., e_{d-1} \}$ are positve integers.
We define $primitive(P,E)$ as a statement that yeilds true
if $p_0^{e_0} p_1^{e_1} ... p_{d-1}^{e_{d-1}}$ is a primitive
abundant number, and false otherwise.

\begin{theorem} 
Let $P$ be a sequence of $d$ odd primes. If $b_{\infty}(P) > 2$, then the 
following algorithm will extract all primitive
abundant odd numbers whose prime divisors compose $P$.
\end{theorem}
% The algorithm:
%
% * Set initial exponents to 1 
%
% * Check if abundant. If so check if primitive
%   * If not primitive, no abundant odds 
%		* If primitive, this is 
%
% * If b(p1^e1 p2^e2 ... pd^ed) >= 2
%		* If primitive this is an option : return true
%		* If not return false
%
%	* Check to see if any exponent increases will cause abundance
%    If any do cause abundance, check if primitive.
%    Always start with the exponents of highest 'influence'
%		 then move on to lower 'influence' exponents. influence
%    is measured by which exponents increase b(n) the most
%    Del_pos is exactly the measure of this.
%
%	* Once the first exponent which does not cause abundance is found
%    we want to check which exponents can be raised to 'infinity'
%    to eventually produce an abundant number. We raise the exponents
%    NEED TO TALK ABOUT ORDER OF INFLUENCE HERE. IN THE ALGORITHM
%    WE ASSUME INFINITE INFLUENCE IS DIRECTLY RELATED TO FINITE
%    SINGLE EXPONENT INFLUENCE? IS THIS TRUE? I DONT KNOW.
%				*Infinite influence: Take a prime-exp pair p^e -> p^infinity
%				*Finite influence  : Take a prime-exp pair p^e -> p^(e+1)
%       Are they related?
% ^ * Mixed b: Raise exponents of low influence pairs to see if they
%     mixed with others will eventually go on to infinity
% ^ * If so raise 

% THIS is going to be UGLY.
% NEW THEorem added - may be important

\textit{In the below algorithm, the sequence of primes P is indexed 
starting at 0, not 1. In other words, $P = \{p_0, p_1, ..., p_{d-1}\}$}

%\subfile{flowcharts/expflow.tex}

{\setlength{\parindent}{0cm}

\textbf{The main algorithm}

\textbf{INPUT:} The sequence of primes 
$P = \{p_0, p_1, ..., p_{d-1}\}$.

\textbf{OUTPUT:} All primitive abundant odds whose prime factors
compose $P$

}


\begin{algorithmic}
%E represents exponents
\IF {$b_{\infty}(P) \leq 2$}
  \RETURN $\{\}$
\ENDIF

\STATE $E := \{e_0 = 1,e_1 = 1,...,e_n = 1\}$
\STATE $A := \{ \}$ %primitive-abunant odds

\STATE $expAbundant(P, E, A)$

\RETURN $A$

\end{algorithmic}

{\setlength{\parindent}{0cm}

\textbf{Algorithm: expAbundant}

\textbf{INPUT:} The sequence of primes 
$P = \{p_0, p_1, ..., p_{d-1}\}$, the exponents to those 
primes $E$, and the primitive abundants found so far, $A$.

\textbf{OUTPUT:} No output. However, $A$ will be modified.
}

\begin{algorithmic}
\IF {$b(P, E) \geq 2$}
	\IF {$primitive(P, E)$}
		\STATE $A := A \& (P,E)$
		\RETURN
	\ENDIF
	\RETURN
\ENDIF
$orderDelPos(P,E,A)$
\RETURN

\end{algorithmic}

{\setlength{\parindent}{0cm}

\textbf{Algorithm: orderDelPos}

\textbf{INPUT:} The sequence of primes 
$P = \{p_0, p_1, ..., p_{d-1}\}$, exponents $E = \{ m_0, m_1,
...\}$, the set of primitive abundant numbers found $A$.

\textbf{OUTPUT:} True or False, depending on if the algorithm found any primitive abundants.
}


\begin{algorithmic}

\STATE $e_{found} :=$ FALSE
%Index of delta_pos in order of 
%of ascending order:
% The last element as the most influence
% G is for generate: Make this function
\STATE $G := G_{ \Delta_{+}}(P,E)$

\STATE $g := |P| - 1$ %Index in I_delta

\STATE $i := 0$ %Prime index

\WHILE {$g \geq 0$}
	\STATE $i := G[g]$

	%\STATE $\Delta_{+} := \Delta_{+}(P_{i+p})$

	\IF {$p_i$ is critical}
		\STATE \COMMENT{$E_{new}$ is the new exponents raising the exponent that
		allows the number to be abundant}
		\STATE $E_{new} := E$
		\STATE $E_{new}[ i_p ] := E_{new}[ i_p ] + 1 $
		
		\IF {$primitive(P, E_{new})$}
			\STATE $A := A\&(P,E)$
			\STATE $e_{found} := $ TRUE
		\ENDIF

		\STATE $g := g - 1$  
	
	\ELSE
		\STATE break;
	\ENDIF
\ENDWHILE


\IF {$i_{\Delta_+} \geq 0$}
	\STATE \COMMENT{Not all indexed exponents
	were able to make an abundant when raised one power}
	\IF {$expIncAlg(P, E, A, G, g)$}
		\STATE $e_{found} := $ TRUE
	\ENDIF
\ENDIF

\RETURN $e_found$

\end{algorithmic}
expIncAlg: Parameters - P, E, A, s

\begin{algorithmic}

\STATE $e_{found} :=$ FALSE
\STATE $g  := s$
\STATE $G = G_{\Delta_+}(P,E)$


\WHILE {$g \geq 0$}
%	\STATE \COMMENT{ exponent indicies that,when increased by 1, 
%do not change the fact that n is deficient}
	\STATE $G_{b < 2} := \{G[0], G[1], 
	                       ..., G[g],\}$

	\IF {$mb(P, E, I) \geq 2$}
		\STATE \COMMENT{When other exponents that would not normally
cause n to be abundant by themselves are increased, this
exponent combined with other exponents in the stated situation
will eventually allow n to be an abundant number. Try increasing
it.}
		\STATE $E_{new} := E$
		
		%Prime index
		\STATE $i_p := G[g]$
		\STATE $E_{new}[i_p] := E_{new}[i_p] + 1$

		% TODO: here we start at i_delt+. How can we assume this?
		% Is it okay to start from the beginning to avoid this assumption?

    % TOO HARD TO PROVE, we shold not assume

		% I think we can assume because if one exponet prime sequence changes,
		% then I shifts over none or by one. If none, we are double
		% checking the prime-exp-pair
		% otherwise should we move back one or is this fine?

		%                                           _ | _
		%                                            \|/
    %                                             .
		\IF {$OrderDelPos(P, E_{new}, A$}%, g)$}
			\STATE $e_{found} :=$ TRUE
		\ENDIF

	\STATE $g :=  g - 1$
	\ENDIF

\ENDWHILE

\RETURN $e_{found}$

\end{algorithmic}


TODO: DISCUSSION ON HOW THIS ALGORITHM WORKS?

\begin{proof}
  In order to prove the theorem we need to show that it
"collects" all odd primitive abundant numbers whose prime factors
compose $P$. Let $P = \{p_0, p_1, ..., p_{d-1}\}$.

  Let $n$ be an odd primitive abundant number whose divisors
compose $P$. We want to show that $n$ is gathered by the
algorithm, as this would prove our theorem. Suppose that $n =
p_0^{m_0}p_1^{m_1} ... p_{d-1}^{m_{d-1}}$. We say that $m = m_0
+ m_1 + ... + m_{d-1}$. 
\\

We will use finite induction for our proof. We want to show that
the algorithm will collect at least one divisors of of order $m -
d, m - d + 1, m - d + 2, ..., 0$ to order $0$.

Our base case is: \textit{The algorithm passes divisor of order $m
- d$.} This is clearly the case as $\ell = p_0 p_1 ... p_{d-1}$ is the 
first integer the algorithm passes, and this divisor has order 
$m-d$.

Our inductive step is:
\textit{Suppose the algorithm collects a divisor of order $k$,
where $0 < k \leq m - d $. We want to show there exists a divisor
of order $k - 1$ that is passed in this algorithm will pass}

We have two cases: There is a special case for $k = 1$, and a
normal case for $k > 1$.

The first appearance of a divisor $\ell$ is where the exponents were
modified and $\ell = (P, E')$ was the result. This will be used
througout the proof.

\textbf{Case 1 ($k > 1$):} Denote our divisor $\ell_k$. Note that
$\ell_k$ can appear for the first time in the algorithm in a few
places: It can appear in $expAbundant$ or $expIncAlg$. Note that
this divisor cannot appear in $orderDelPos$, for if $\ell_k$ first
appeared here, the conditions for $\ell_k$ to first appear here
would be such that $\ell_k$ is abundant.

Suppose that $\ell_k$ first appears in $expAbundant$. Note that, as
$k \neq 1$, we have that $b(\ell_k) < 2$. Thus the algorithm moves on
to $orderDelPos(P, E_k, A)$.
\\

Suppose that $\ell_k$ first appears in $expIncAlg$. Notice that the
algorithm will immediently check $orderDelPos(P, E_k, A)$.
\\

In both cases, we will have that $orderDelPos$ is called. In
$orderDelPos$, notice that $G = (P,E_k)$ is formed. Notice that $G$
is in the form such that if $i$ appears before $j$ in $G$, we have
that $\Delta_+(p_i, e_i) \leq \Delta_+(p_i)$. And therefore, we
have that $\Delta_+(p_i, e_i) * b(\ell_k) < \Delta_+(p_j,e_j)$.
Notice that if $\Delta_+(p_i,e_i) * b(\ell_k) \geq 2$, then $p_i$ is
a crucial prime. In fact, if $p_i$ is a crucial prime, then $p_j$
is a crucial prime.

It is also important to note that no incomplete prime(in $\ell_k$)is a
crucial prime. A multiplication by any incomplete prim of $\ell_k$ 
will result in another divisor of $n$, this divisor will have order $k
- 1$. As $k > 1$, we have that $k - 1 > 0$, and hence our new
  divisor $\ell_{k-1}$ will have nonzero order, therefore it will be
a proper divisor of $n$. Thus, $\ell_{k-1}$ is not an abundant number
and therefore our incomplete prime was not a crucial prime.

This, $G(P,E_k)$ is sorted in a way where all crucial primes are on
the right, and all incomplete primes are on the left. 

$$ G = \{ \text{other indicies and incomplete indicies ... $|$ crucial
indicies...} \} $$

This allows us to conclude a few things. First of all,  there
exists a non-crucial prime index in $G$ (as there exist at
least one incomplete prime in $\ell_k$). Secondly, the loop in
$orderDelPos$ will only terminate once every crucial prime index of
$G$ has been through the loop and none of the incomplete prime
indicies have been checked. When the loop terminates, the 
algorithm will call $expIncAlg(P,E_v, A, g)$.
\\

In $expIncAlg(P,E_v,A, g)$, notice that $G_{b<2}$ will be formed
from the remains of $G$, i.e. $G_{b<2} = \{G[0], G[1], ...,
G[g]\}$. Notice that $G_{b<2}$ contains all incomplete primes
of $\ell_k$. Keep in mind that these primes exponents are with
respect to our primitive abundant number $n$. Thus, when the
incomplete exponents of each incomplete prime is raised high
enough,  the resutlt will be $n$. Thus the limit function
$mb(P,E_k, G_{b<2})$ is evaluated, the result is guaranteed to be
greater than two. Thus we will descend into the while loop of
$expIncAlg$. 

Let $i = G[g]$. Follow the while loop until $p_i$ is incomplete in
$\ell_k$. When this is the case, we know that a divisor of order $k -
1$ will appear in the form $p_i \ell_k = \ell_{k-1} = (P, E'_k)$ when
$e_i$ is increased. Thus the inductive step has been proven.

\textbf{Case 2 ($k = 1$):} In this case, we note that there is
only one incomplete prime and it must be a crucial prime.
Since $\ell_1$ can only be discovered in $expAbundnat$ or
$expIncAlg$, and both lead to $orderDelPos(P, E_1, A)$, we will be in
the loop of $orderDelPos$. As all crucial prime indicies will pass
the if statement in the loop, we have that our incomplete prime's
exponent will be increased and the result will be $n$. Thus the
inductive step has been proven.
\\

In both cases, the inductive step was prove, and therefore by
finite induction, the algorithm has retrieved $n$.

\end{proof}

\subsection{Properties of the Exponent Algorithm}

\begin{theorem}[Property 1]

\end{theorem}

\begin{conj}[Continuity of Primative Abundant Numbers]:
Let $q \leq r$ be primes. Suppose that $p_0, p_1, p_2, ..., 
p_{l-2}, q$ are the first l divisors of some primitive abundant
odd with $d$ prime divisors. Suppose $p_0, p_1, ..., p_{l-2},r$ 
are also the first $l$ divisors of some primitive abundant odd
with $d$ prime divisors. Then, given any prime $s$ such that $q \leq s 
\leq r$, we have that $p_0, p_1,..., p_{l-2}, s$ are the first $l$
divisors of some primitive abundant odd with $d$ divisors.
\end{conj}

THIS PROPERTY IS FALSE.
\\

The below paragraphs are in construction. Be warned!!
\\

%NOTE THAT $3 * 5^6 * 7^3 * 67 * 73$ IS PRIMITIVE ABUNDANT,
%AND $3^2 * 5 * 7 * 107 * 109$ IS PRIMITIVE ABUNDANT,
%YET NO PRIMITIVE ABUNDANT WITH 5 DIVISORS EXISTS SUCH THAT
%ITS FIRST FOUR DIVISORS ARE $3,5,7$ AND $71$

%SOURCE: VAKDAS DICIUNAS,
%$https://klevas.mif.vu.lt/~valdas/Abundant_Numbers/$

%Notice that we are not far off. We are only missing about
%2 percent. And the exponent algorithm is in good condition.
%Great sifting assumption.


%IDEA IDEA IDEA!!
%Suppose that $p_1 ... p_{i-1} q$ fails in the opm algorithm. IDEA:
%We should find the maximum prime $q'$ such that $$p_1 ... p_{i-1}
%q' \in opm(i)$$. T
%hen we can start at that prime!!(Maybe )
%(Possibly)

%Why this idea?? 

Consider the example above, where $\{3,5,7,73\}$ fails.
Note that $\{3,5,7,73\} \in opm(4)$. In fact, \textit{the first
prime after q' = 103 is 107}, and note that this prime will form a
primitive abundant number in opm(5) with $3,5,7$. Thats the
conjecture.


The idea is that if the primest does not appear in opm(i), then it
is not going to create an abundant number without another prime in
d > i. We also may be able to prove that the last few primes q' in
opm(i) are going to create nonprimitive abundants.
\\

%CITE VALDAS for opm - odd primitive abundants.

Here is an important theorem

\textit{Suppose that $p_0, p_1, p_2, ..., 
p_{d-2}, q$ are the divisors of some primitive abundant
odd with $d$ prime divisors. Suppose $p_0, p_1, ..., p_{d-2},r$ 
are also the divisors of some primitive abundant odd
with $d$ prime divisors. Then, given any prime $s$ such that $q \leq s 
\leq r$, we have that $p_0, p_1,..., p_{d-2}, s$ are the 
divisors of some primitive abundant odd with $d$ divisors.}

Note that since each sequence
of primes will produce a primitive abundant number, we have that
the exponent algorithm will produce all possible combinations of
exponents that create abundant numbers for the prime sequences $Q = 
\{p_0, p_1, ..., p_{d-2}, q\}$ and $R = \{p_0, p_1, ..., p_{d-2},
r\}$. Also, notice that both Q and R will have the same initial
$G_{\Delta_+}$ values when used with the initial exponent sequence
in the exponent algorithm. And as the primes are in increasing
order, the $G_{\Delta_+}$ values are going to be exactly the same
for Q and R, and in fact (giving q and r indicies of d-1) we have
that $G_{\Delta_+}(Q, \{1,1,1,...\}) = G_{\Delta_+}(R,
\{1,1,1,...\}) = \{(d-1), (d-2), ..., 1, 0\}$. Because of this,
the exponent algorithm will follow the same path for a while for
both $Q$ and $R$.

Suppose first that $Q$ and $R$ both hit an abundant number without
diverging. Then using the same exponent set, $S$ will be
primitive.

\textbf{CASE 2:}

Now suppose that $Q$ hits an abundant and $R$ has not, yet the
algorithm has not diverged. 

\textbf{CASE 2.1}

Suppose first that $S$ is abundant
with exponents generated by $Q$'s algorithm path. Then $S$ will be
primitive with the exponents generated, for if not, we have that
a divisor of $S,E$ is going to be primitive abundant. We do not
care if this divsor shares the same prime factors, we only care if
the primitive abundant divisors are missing a prime factor: In
this case, we will have that $b(d = p_0^{m_0} ... p_{i-1}^{m_{i-1}}
p_{i + 1}^{m_{i+1}}... p_{d-2}^{m_{d-2}} s^{m_{d-1}}) > 2$,
however, it is easy to show that this divisor $d$ is such that
$b(d) < b(p_0^{m_0} p_1^{m_1} ... p_{d-2}^{m_{d-2}}) < 2$, as the
entry on the right of the equality is a divisor of our primitive
abundant odd generated by $Q,E$. Thus $S,E$ (or one of its
divisors) is primitive abundant number(with primes composing $P$).

\textbf{CASE 2.2}

%THIS CASE ALSO COVERS WHEN $Q$ AND $R$ DIVERGE EARLY, SO THE BASE
%CASE SHOULD BE DONE!

What would happen if $S,E$ is not abundant when $Q,E$ is?

Set $E$ to be the exponent set of the primitive abundant number
whose prime divisors compose $R$.

Suppose that $S,E$ is an abundant number but is not primitive and
none of its divisors who lose no prime factors are primitive. Then
$S,E$ has abundant divisor $d = p_0^{m_0} ... p_{i-1}^{m_{i-1}}
p_{i+1}^{m_{i+1}} ... s^{m_{d-1}}$. Note that this divisor has the
following property:

$$b(d) < b(p_0^{m_0} p_1^{m_1} ... p_{d-1}^{m_{d-1}}) < 2$$

And this is the case as the divisor in the middle is a proper
divisor of primitive abundant $R,E$.


%% CONJECTURE: FIRST FIND IN THE ALGORITHM WILL NOT BE TURNED DOWN 

\end{document}
