\documentclass[../paper.tex]{subfiles}

\begin{document}

\section{Exponent Algorithms}

\textbf{Theorem:} Let $P$ be a set of $d$ odd primes. If $b_{1}(P) \leq 2$
and $b_{\infty}(P) > 2$, then the following algorithm will extract all primative
abundant odd numbers whose prime divisors compose $P$.

% The algorithm:
%
% * Set initial exponents to 1 
%
% * Check if abundant. If so check if primative
%   * If not primative, no abundant odds 
%		* If primative, this is 
%
% * If b(p1^e1 p2^e2 ... pd^ed) >= 2
%		* If primative this is an option : return true
%		* If not return false
%
%	* Check to see if any exponent increases will cause abundance
%    If any do cause abundance, check if primative.
%    Always start with the exponents of highest 'influence'
%		 then move on to lower 'influence' exponents. influence
%    is measured by which exponents increase b(n) the most
%    Del_pos is exactly the measure of this.
%
%	* Once the first exponent which does not cause abundance is found
%    we want to check which exponents can be raised to 'infinity'
%    to eventually produce an abundant number. We raise the exponents
%    NEED TO TALK ABOUT ORDER OF INFLUENCE HERE. IN THE ALGORITHM
%    WE ASSUME INFINITE INFLUENCE IS DIRECTLY RELATED TO FINITE
%    SINGLE EXPONENT INFLUENCE? IS THIS TRUE? I DONT KNOW.
%				*Infinite influence: Take a prime-exp pair p^e -> p^infinity
%				*Finite influence  : Take a prime-exp pair p^e -> p^(e+1)
%       Are they related?
% ^ * Mixed b: Raise exponents of low influence pairs to see if they
%     mixed with others will eventually go on to infinity
% ^ * If so raise 

% THIS is going to be UGLY.
% NEW THEorem added - may be important

\begin{algorithmic}
%E represents exponents
\STATE $E := \{e_1 = 1,e_2 = 1,...,e_n = 1\}$
\STATE $A := \{ \}$ %primative-abunant odds

\STATE $expAbundant(P, E, A)$

\RETURN $A$

\end{algorithmic}

expAbundant:

\begin{algorithmic}
\IF $b(P, E) \geq 2$
	\IF $primative(P, E)$
		\STATE $A := A & (P,E)$
		\RETURN TRUE
	\ENDIF
	\RETURN FALSE
\ENDIF

\RETURN $orderDelPos(P,E,A,|P| - 1)$

\end{algorithmic}

orderDelPos:
Parameters: pset, expset, A, startat)

\begin{algorithmic}

\STATE $e_found :=$ TRUE
%Index of delta_pos in order of 
%of ascending order:
% The last element as the most influence
% G is for generate: Make this function
\STATE $I_{\Delta_{+}} := G{ \Delta_{+}}(P,E)$

\STATE $i_{\Delta_{+}} := s$ %Index in I_delta
\STATE $i_p := 0$ %Prime index





\end{algorithmic}
\end{document}
