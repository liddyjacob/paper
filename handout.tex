
\documentclass[12pt]{article}

\usepackage{amsmath}
\usepackage{amsfonts}
\usepackage{amsthm}
\usepackage{amssymb}
\usepackage{enumitem}
\usepackage{algorithmic}

\usepackage[scale=0.77,a4paper,margin=.75in]{geometry}

\newtheorem{theorem}{Theorem}
\newtheorem{conj}{Conjecture}
\newtheorem{coro}{Corollary}
\newtheorem{lem}[theorem]{Lemma}
%\numberwithin{theorem}{section}
%\numberwithin{coro}{section}
%\numberwithin{lemma}{section}

%opening
\title{HANDOUT}
\begin{document}

\maketitle

\Large  
\textbf{Types of Integers: } Let $n$ be a positive integer.

We say that $n$ is \textbf{perfect} if $n$ is equal to the sum of its \textit{proper} divisors, alternatively  $2n$ is equal to the sum of its divisors($\sigma(n) = 2n$).
\vspace{-3mm}

We say that $n$ is \textbf{abundant} if $2n$ is less than or equal to the sum of its proper divisors. ($\sigma(n) \geq 2n$) .
\vspace{3mm}

We say that $n$ is \textbf{deficient} if $n$ is not abundant($\sigma(n) < 2$).
\vspace{3mm}

We say that $n$ is \textbf{primitive abundant} if $n$ is abundant and all proper divisors of $n$ are deficient.
\vspace{3mm}

We say that $n$ is \textbf{semiperfect}, or \textbf{pseudoperfect} if $n$ is abundant and there exists a subset $S$ of the divisors of $n$ such that $n = \sum_{s \in S} s$.
\vspace{3mm}

We say that $n$ is \textbf{weird} if $n$ is abundant but $n$ is not semiperfect. 

\textbf{Functions: } Suppose $n$ is a positive integer with prime 
factorization $n = p_1^{m_1} p_2^{m_2} ... p_d^{m_d}$.
Let $P = \{p_1, p_2, ..., p_d\}$, and $E = \{m_1, m_2, ..., m_d\}$. 
Suppose $p$ is a prime.
\vspace{3mm}

$\sigma(n) = \sum_{d | n}(d) =  
\sigma(p_1^{m_1}) \sigma(p_2^{m_2}) ... \sigma(p_d^{m_d}) = 
\prod_{1 \leq i \leq d} \frac{p_i^{m_i + 1} -1}{p_i -1}$
\vspace{3mm}

$b(n) = \frac{\sigma(n)}{n} = \prod_{1 \leq i \leq d} \frac{p_i^{m_i + 1} -1}{p_i^{m_i}(p_i -1)}$
\vspace{3mm}

$b_{\infty}(p) = \lim_{m \rightarrow \infty} b(p^m) = p / (p - 1)$
\vspace{3mm}

$  b_{\infty}(P) 
  = \prod_{1 \leq i \leq d} \lim_{m \rightarrow \infty}b(p_i^m)
                = \prod_{1 \leq i \leq d}(\frac{p_i}{p_i - 1})
$
\vspace{3mm}

$\Delta_+(p,m) = \frac{p^{m+2} - 1}{ p (p^{m + 1} -
1)}$; $ b(n) \Delta_+(p_i, m_i) = b(p_i n)$



\end{document}
