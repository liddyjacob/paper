\documentclass[../paper.tex]{subfiles}


\begin{document}

%TODO: Mention that p_1, p_2, ..., p_d are always listed in increasing 
% order unless said not to be.
\section{The Algorithm}

\subsection{The Primative Abundant Tree Structure}

We want to construct a useful structure that relates the primative
abundant odd prime factors. 

Let $\mathbf{P}$ be a set of 'nodes' such that each $n \in 
\mathbf{P}$ is defined as a pair including a prime greater 
than or equal to 3 and a set of primes before it. $n$ takes the 
form $(p4, \{p1, p2, p3\})$. Examle: $(11, \{3, 5\})$.
$n$ can also be in the form $(1, \{\})$

Let $V$ be a set of pairs of nodes, called edges.

Finally, we define an acyclic connected graph $G_d(\mathbf{P},V)$ 
for some positive integer $d \geq 3$ where $G_d$ satisfies the 
following rules: 

\begin{enumerate}

\item $(1, \{\}) \in \mathbf{P}$. Call this node $n_0$ for the next 
      statement.

\item If $p_1^{m_1}p_2^{m_2}...p_d^{m_d}$ is an primative abundant 
      odd for positive integers $\{m_1, m_2, ..., m_d\}$, then
			the following are true: 

      i. $ (p_1, \{\}), (p_2, \{p_1\}), (p_3, \{p_2, p_1\}), ..., 
        (p_d, \{p_1, p_2, p_3, ..., p_{d-1}\}) \in \mathbf{P} $
			Call these nodes $n_1, n_2, ..., n_d$ for simplicity in the 
			next statement.

      ii. $ (n_0, n_1), (n_1, n_2), (n_2, n_3), ..., (n_{d-1}, n_{d}) 
      \in V $

\end{enumerate}

Note that this graph contains a finite number of nodes since 
there are a finite number of primative odd abundant numbers with
d divisors. 

When a diagram is drawn, the nodes are represented by the prime
value they hold, and the set of primes before it can be deduced
from the edges in the graph. See the diagram of $G_3$ and $G_4$
below.

\subsection{Modification for the Algorithm}

We need to associate two statements with each node for the 
algorithm.

We define the statement $E_{(d,P)}$ for some set of primes P such 
that $E_{(d,P)} = $ "There exists a node $n \in G_d$ such that the set 
of primes $P$ is associated with $n$". In other words this state that 
there exists some primative abundant with $d$ divisors whose prime
factors are $P$.

We also define the 'exhausted' statement $X_{(d,P)}$ for some set of 
primes $P = \{p_1, p_2, ..., p_i\}$   by $X_{(d,P)} = $ "all primes 
$q_{i+1}, q_{i+2}, ..., q_d$ have been found such that $p_1,p_2, 
..., p_i, q_{i+1}, q_{i+2}, ...q_d$ are all the prime factors to some
primative abundant odd". This statement is specific to the algorithm 
and when a new node is created in the algorithm this will be included
as a third paramater that will be false by default. 


\subsection{The Algorithm}
\textbf{Algorithm} 

Consider $G_d$. We want to determine what nodes are in $G_d$, 
which will eventually allow us to determine the primative abundant 
odds with $d$ divisors.

We will allow the nodes in $\mathbf{P}$ to have a boolean 
attribute to them. This boolean will represent the statement in 
the previous section known as $X_{(d,P)}$. Thus a node of 
$\mathbf{P}$ will be in the form 
$(p_i, \{p_1, p_2, ..., p_{i-1}\}, true/false)$.

\noindent
We define a few simple functions here:

Let $n \in \mathbf{P}$.
%
We define $n.P()$ will retrieve the prime associated with n, as
well as the set of primes associated with n, returning them as a 
sequence, from least to greatest prime.

\noindent
Now we define a few more advanced functions in the algorithm 
notation.


Let $G = (\mathbf{P}, V)$, and
let $n \in \mathbf{P}$ such that n has no child nodes. 
%Todo: Define a child node
%
We define $delete(n, G)$ in the following way.  




let 

\begin {algorithmic}
\STATE TODO: 

\end {algorithmic}
% Use this to declair new algorithms.
{\setlength{\parindent}{0cm}

\textbf{Algorithm: ExtractPrimativeAbundantOdds}

\textbf{INPUT:} d, desired divisors.

\textbf{OUTPUT:} All Primative Abundant Odds with d divisors.

}


{\setlength{\parindent}{0cm}

\textbf{Algorithm: ExtractPrimativeAbundantOdds}

\textbf{INPUT:} d, desired divisors.

\textbf{OUTPUT:} All Primative Abundant Odds with d divisors.

}

\begin{algorithmic}

\STATE $\mathbf{P} := \{(1, \{\}, FALSE)\}$
\STATE $V := \{\}$
\STATE $G_d := (\mathbf{P},V)$
% |P| = i
% NOTE: a LOT of graph and algorithm notation is borrowed from
%the book INTRODUCTION TO ALGORITHMS, THIRD EDITON (Cormen)(2011)
%WE NEED TO CITE OUR SOURCE, so people will know where to look to
%understand the notation here.
\WHILE {$(1, \{\}, FALSE) \in \mathbf(P)$}
	\STATE $newnode = Insert(p,\mathbf{P})$ 
	%TODO define insert - insert element into list
	\STATE $P = newnode.P()$
	\STATE


	\IF {$b(p_1 p_2 p_3 ... p_i) \geq 2$}	
		\IF {$i = d$}
			\STATE $success(newnode)$
		\ELSE
			%TODO: Define deficient prime == ALSO DETAILS ABOUT EDGES@!!!1!!!!!
			% REALLY IMPORTANT THAT YOU TALK ABOUT NOT CHANGING CONNECTIONS
			% IMPORTANT THAT IF (11, ...) -> (17, ...) AND 17
			% IS CHANGED TO 19 THAT THE CONNECTION REMAINS!!
			% (11, ...) -> (19, ...);
			\STATE $deficientNode(newnode)$
			\STATE $P \leftarrow newnode.P()$
		\ENDIF
	\ENDIF
	\IF{$b_{\infty}(p_0, p_2, ..., p_i) > 2$}
		\IF {$i == d$}
			\STATE $fail(newnode)$
		\ENDIF
	\ELSIF {$i == d$}
		%TODO: define this function.
		\IF {$expFind(P)$}
			\STATE $success(newnode)$
		\ELSE
			\STATE $fail(newnode)$
		\ENDIF
	\ENDIF

\ENDWHILE
% What this does is checks if the element FAlSE in the initial
% node that includes 1 as it's 'prime' has not changed its 
% boolean value 'exhausted' to True.
  %\WHILE {$(1, \{\}, FALSE) \in \mathbf{P}$ do}
	%	\STATE test
	%
%
\end{algorithmic}

\textbf{END}


ALGORITHM:



\textit{Proof of correctness of the algorithm:}
%Todo : Start writing up algorithm in notebook.
We first start by considering 



\end{document}
