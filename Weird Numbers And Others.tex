\documentclass[]{article}

\usepackage{amsmath}
\usepackage{amsfonts}
\usepackage{amsthm}
\usepackage{amssymb}
\usepackage{enumitem}   

%opening
\title{An Upper Bound for the number of prime divisors of an odd weird number}
\author{Jacob P. Liddy}

\begin{document}

\maketitle

\begin{abstract}
A primative abundant number is a number such that none of it's divisors (other than itself) are abundant numbers. Dickson proved that there exists a finite number of primative odd abundant numbers with d prime divisors. In this paper we find an algorithm that finds all primative odd numbers with d divisors, and use that algorithm to prove that an odd weird number must have at least 5 prime divisors.
%TODO: can we get above five divisors
\end{abstract}

\section{Introduction}
We define the function $\sigma(n):\mathbb{N} \rightarrow \mathbb{N}$
$$\sigma(n) = \sum_{d|n}d.$$
the sum taken over the divisors of $n$. We say $n$ is \textit{abundant} if $\sigma(n) > 2n$, \textit{perfect} if $\sigma(n) = 2n$, and \textit{deficient} if $\sigma(n) < 2n$ The proper divisors(Or aliquot parts) of $n$ are the positive divisors of $n$ that are less than $n$. We may denote this set by
$$\textit{A}_{n} = \{d \mid 0 < d < n\}$$
We say that $n$ is \textit{pseudoperfect} if $n$ is abundant and there exists a subset $S \subset \textit{A}_{n} $ such that
$$ n = \sum_{d \in S} d$$
We say that $n$ is \textit{weird} if $n$ is abundant but not pseudoperfect. The smallest weird number is 70, and all known weird numbers are even.


%Interesting fact- finite number of primative odds with d divisors (not just prime divisors)

Dickson proved that a primative odd number must have  
\section{Our Preliminaries}

We define $b(n)$ by
$$b(n) = \frac{\sigma(n)}{n}$$
Hence $n$ is abundant if and only if $b(n) > 2$

Let $d_i d_j = n$, where $d_{i}$ and $d_{j}$ are divisors of $n$. This implies that 
$$\frac{d_i}{n} = \frac{1}{d_j}$$
Note that $\sigma(n) = \sum_{d_{i}|n} d_{i}$. Hence,
	\begin{align*}
		b(n) = \sum_{d_{i}|n} \frac{d_{i}}{n} &= \sum_{d_{j}|n} \frac{1}{d_{j}}\\
		&= \sum_{d|n} \frac{1}{d}
	\end{align*}
Let $n$ be written in its prime factorization. Then $n=p_1^{m_1}p_2^{m_2}p_3^{m_3}...p_{k}^{m_k}$ It can be shown that $$\sigma(n) = \prod_{1 \leq i \leq k} \frac{p_i^{m_i + 1} -1}{p_i -1}$$
Note that $n = p_1^{m_1}p_2^{m_2}p_3^{m_3}...p_k^{m_k} = \prod_{1 \leq i \leq k} p_i^{m_i}$. Thus, $b(n)$ can be expressed as:
$$b(n) = \frac{\alpha(n)}{n} = \prod_{1 \leq i \leq k} \frac{p_i^{m_i + 1} -1}{(p_i -1)p_{i}^{m_{i}}}$$

\section*{Lemmas And Theorems}\par\hspace{4ex}

\textbf{Lemma A:} All primative non-deficient odd numbers having a given number $d$ of distinct prime factors are formed from a finite number of sets of $d$ primes.

\par
\textbf{Lemma B:} Let $d$ be a desired number of prime divisors. Let $P$ be a set of $k$ prime numbers, where $k < d$. Let $p_0$ be a prime greater than the largest prime divisor in $P$. Generate the next $d - k$ sequential primes after $p$, and call these primes $p_1, p_2, ..., p_{d-k}$ respectivly. If the follwing equality is true, then $P \cup  {p_0, p_1, p_2}...$ can form an abundant number.



$$ \sum_{p \in P \cup {p_0, p_1, p_2}} (\frac{p+1}{p}) \geq 2 $$

\textbf()

\section{The Algorithm}

\subsection{The Primative Abundant Tree Structure}

We want to construct a useful structure that relates the primative
abundant odd prime factors. Let $\mathbf{P}$ be a set of 'nodes' 
such that each $n \in P $

Consider an acyclic connected graph $G_d(\mathbf{P},V)$ such that 

\subsection{Modification for the Algorithm}

\subsection{The Algorithm}
\textbf{Algorithm} 

\textbf{Corollary to lemma B:} If the equality is true, then $n^{m_1}_{1} n^{m_2}_{2} ... n^{m} $


\textbf{Theorem:} All odd primative abundant numbers with d divisors will be obtained through the following algorithm:

Let $P_n = \{\}$ 


\begin{enumerate}[label=(\Roman*)]
\item Find the next odd prime $p_0$ greater than the largest prime in $P_n$. If the current primeset $P_n => P_n \cup \{p\}$ goes on to produce a primative abundant number, then try the next prime after $p_0$, denoted $p_1$. Likewise if $P_n => P_n \cup p_1$ goes on to produce a primative abundant, then try $p_2, p_3, ... $ until one prime fails. 

%NOTE: ^ PROVE THIS ENDS WITH THE THEOREM FROM EUGINE DICKSON

\item Check to see if the set of primes currently can form an abundant number(Using $b_{\infty} (P_n) > 2$). If this is the case, move on to (III). Otherwise move back to (I).

\item Let ${p_1, p_2, p_3, ... , p_k} = P_n$. Check to see if $n = p_1 p_2 p_3 ... p_k$ is abundant. 

\begin{enumerate}[label=(\roman*)]
  \item If it is, and $k = d$, then $n$ is a primative odd abundant number with $d$ prime divisors, as desired. If $k < d$, then $n$ is a primative abundant number with $k$ divisors, which is not desired.
  \item If it is not abundant, then move on to (IV) 
\end{enumerate}


\item If the number of primes $k$ is less than $d$, then we move back to (I). If it is equal to $d$, then we must find the right exponents for $P_n$, by moving on to (V) 
\item EXPONENTS

\end{enumerate}




\section{Other Paper's Preliminaries}

Consider the finite set $S \subset \mathbb{N}$ given by
$$S = \{a_1,a_2,a_3,...,a_n\}$$.
We define$\sum(S)$ by
$$\sum(S) = \{\sum_{j=1}^n \epsilon_j a_j : \epsilon_j = 0 or 1, 1 \leq j \leq n \}.$$
That is, $\sum(S)$ is the set of all possible sums taken from the elements of S, if no element is taken more than once. Clearly $|\sum (S)| \leq 2^n$.

Thus $n$ is pseudoperfect if and only if $n$ is abundant and $n \in \sum (\textit{A}_n)$. We define
$$a(n) = \sigma(n) - 2n$$,
hence n is abundant if and only if $a(n) > 0$. Since
$$a(n) + n = \sigma (n) - n = \sum_{d\in \textit{A}_n}d,$$
it follows immediately that an abundant number $n$ is pseudoperfect if and only if $a(n) \in \sum(\textit{A}_n)$. 



\end{document}
