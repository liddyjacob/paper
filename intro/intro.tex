\documentclass[../paper.tex]{article}


\begin{document}

\begin{abstract}
An abundant number is said to be primitive if none of its proper 
divisors are abundant.  Dickson proved that for an arbitrary
positive integer $d$ there exists only finitely many odd primative
abundant numbers having exactly $d$ prime divisors\cite{dickson}.
In this paper we describe a fast algorithms
that find all primitive odd numbers with d divisors, and use that
algorithm to prove that odd weird and odd perfect numbers must
have at least six prime divisors.
%TODO: can we get above five divisors
\end{abstract}

\section{Introduction}
The proper divisors of a positive integer $n$ are the positive
divisors of $n$ that are less than $n$. We denote the set of 
proper divisors of n by $\textit{A}_{n}$.

We define the function 
%
$\sigma: \mathbb{N} \rightarrow \mathbb{N}$
%
such that
%
$$\sigma(n) = \sum_{d|n}d$$
%
the sum taken over the divisors of $n$. We say $n$ is 
\textit{abundant} if $\sigma(n) \geq 2n$ \footnotemark
, \textit{perfect} if $\sigma(n) = 2n$, and \textit{deficient} if 
$\sigma(n) < 2n$. We say that $n$ is \textit{pseudoperfect} if 
$n$ is a non-perfect abundant and there exists a subset $ S 
\subset \textit{A}_{n}$ such that
%
\footnotetext{This is the definition Erd\H{o}s gave for abundant
numbers.\cite{erdos}}
%
%
$$ n = \sum_{d \in S} d$$
%
We say that $n$ is \textit{weird} if $n$ is abundant but not 
pseudoperfect or perfect. The smallest weird number is 70, and 
all known weird numbers are even.

If $n$ is abundant and none of its proper divisors are abundant,
we say that $n$ is a primitive abundant number. This paper focuses
on odd primitive abundant numbers, which we refer to as OPNs
(Following the notation of \cite{valdas} and \cite{amato})

It is not known whether any odd perfect or odd weird numbers
exists. However, it is easy to show that any odd perfect number is
an OPN, if such a number exists. Likewise if an odd weird number
$w$ exists, it can be shown that $w$ is an OPN or $w$ is the
multiple of a weird OPN.

Suppose we are able to find all OPNs with $d$ or less prime divisors, 
and each of those OPNs are semiperfect. Then we know that an odd 
weird number an odd perfect number must have at least $d + 1$ prime 
divisors, if such numbers exist.

Therefore a reasonable approach would be to find an algorithm to
determine all odd primitive abundant numbers with a given number 
of unique prime divisors, and then determine if any of those OPNs
are weird or perfect. Such an algorithm will find one of the
desired numbers, or impose a condition on the existance of odd
weird and odd perfect numbers. In this paper, we follow the
approach and find such an algorithm. We also prove that the
algorithm does indeed find all OPNs with $d$ divisors, for any $d
\geq 3$.

If we can get this algorithm to run quickly enough, we will be
able to determine the number of odd primitive abundant numbers
with $6$ divisors, a currently unsolved problem\cite{amato}.
\end{document}
