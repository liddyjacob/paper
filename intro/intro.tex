\documentclass[../paper.tex]{article}
\begin{document}

\begin{abstract}
For an integer $n$, if the sum of the proper divisors of 
$n$ equals or exceeds $n$, then we say that $n$ is an 
\textit{abundant number}.  An abundant number is said to be 
\textit{primitive} if none of its proper divisors are abundant. 
By definition, an abundant number must have at least one primitive 
abundant divisor. In 1913, Dickson proved the following: 
for an arbitrary positive integer $d$ there exists only finitely 
many odd primitive abundant numbers having exactly $d$ distinct 
prime divisors. In 2017, all odd primitive abundant numbers with 
up to $5$ distinct prime divisors have been found by Di\v{c}i\={u}nas. 
In this paper, we 
describe a fast algorithm that finds all odd primitive abundant 
numbers with $d$ distinct prime divisors. We use this algorithm 
to find all odd primitive abundant numbers with $6$ distinct prime
divisors. An abundant number $n$ is said to be \textit{weird} if 
no subset of the proper divisors of $n$ sums to $n$. We use our 
algorithm to show that an odd weird number must have at least $6$
prime divisors.
\end{abstract}

\section{Introduction}
The proper divisors of a positive integer $n$ is the set of 
positive divisors of $n$ excluding $n$. We denote 
the set of proper divisors of $n$ by $\textit{A}_{n}$.

We define the function 
%
$\sigma: \mathbb{N} \rightarrow \mathbb{N}$
%
such that
%
$$\sigma(n) = \sum_{d|n}d$$
%
the sum taken over the divisors of $n$. We say $n$ is 
\textit{abundant} if $\sigma(n) \geq 2n$ \footnotemark
, \textit{perfect} if $\sigma(n) = 2n$, and \textit{deficient} if 
$\sigma(n) < 2n$. We say that $n$ is \textit{pseudoperfect} if 
$n$ is a non-perfect abundant and there exists a subset $ S 
\subset \textit{A}_{n}$ such that
%
\footnotetext{This is the definition Erd\H{o}s gave for abundant
numbers.\cite{erdos}}
%
%
$$ n = \sum_{d \in S} d .$$
%
We say that $n$ is \textit{weird} if $n$ is abundant but not 
pseudoperfect or perfect. The smallest weird number is 70, and 
all known weird numbers are even.

If $n$ is abundant and all of its proper divisors are deficient,
we say that $n$ is a \textit{primitive abundant number.} 
An abundant number must 
have at least one primitive abundant divisor. By our definition,
the smallest primitive abundant number is 6. Some authors exclude perfect
numbers from abundant numbers, in which case the smallest primitive abundant
number is 20. This paper focuses
on odd primitive abundant numbers, which we refer to as OPANs
(Following the notation of \cite{valdas} and \cite{amato}). The sequence 
of OPANs is available at \url{https://oeis.org/A006038}.\cite{oeis}

It is not known whether any odd perfect or odd weird numbers
exists. However, it is easy to show that any odd perfect number is
an OPAN, if such a number exists. Likewise if an odd weird number
$w$ exists, it can be shown that $w$ is an OPAN or $w$ is the
multiple of a weird OPAN.

Our purpose is to find an algorithm that enumerates all OPANs
with a fixed number of unique prime divisors. With this, we
classify all OPANs with $5$ prime divisors as weird, perfect, or pseudoperfect.
We found that all OPANs with $3, 4$, or $5$ divisors are pseudoperfect,
which allows us to conclude that an odd weird number and an odd perfect number must have at least $6$ unique prime factors.

In this paper, we succeed in finding the described algorithm. We 
find all odd primitive numbers with $6$ prime divisors or less. 
Finding all OPANs with $6$ prime divisors 
was previously an unsolved problem\cite{amato}.

Through implementing the algorithm in Section 5, we were able to find the
number of OPANs with 6 divisors. We found that $|OPAN(6)| = 59687996404445$


The largest odd primitive abundant number with 6 divisors is
$3^{38} 5^{28} 17^{16} 257^8 65537^4 4294967291^2$. It has 116 digits.



%Suppose we are able to find all OPANs with $d$ or fewer prime divisors, 
%and each of those OPANs are semiperfect. Then we know that an odd 
%weird number an odd perfect number must have at least $d + 1$ prime 
%divisors, if such numbers exist.

%Therefore a reasonable approach would be to find an algorithm to
%determine all odd primitive abundant numbers with a given number 
%of unique prime divisors, and then determine if any of those OPANs
%are weird or perfect. Such an algorithm will find one of the
%desired numbers, or impose a condition on the existance of odd
%weird and odd perfect numbers.

%If we can get this algorithm to run quickly enough, we will be
%able to determine the number of odd primitive abundant numbers
%with $6$ divisors, a currently unsolved problem\
\end{document}
