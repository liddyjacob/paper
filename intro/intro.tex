\documentclass[../paper.tex]{article}
\begin{document}

\begin{abstract}
An abundant number is said to be primitive if none of its proper 
divisors are abundant.  Dickson proved that for an arbitrary
positive integer $d$ there exists only finitely many odd primitive
abundant numbers having exactly $d$ prime divisors\cite{dickson}.
In this paper we describe a fast algorithm
that finds all primitive odd numbers with d unique prime divisors. 
We use this algorithm to find all odd primitive 
abundant numbers with 6 unique prime divisors. We also conclude that an 
odd weird number has at least 6 unique prime divisors.
%and use that
%algorithm to prove that odd weird and odd perfect numbers must
%have at least six prime divisors.
%TODO: can we get above five divisors
\end{abstract}

\section{Introduction}
The proper divisors of a positive integer $n$ are the positive
divisors of $n$ that are less than $n$. We denote the set of 
proper divisors of n by $\textit{A}_{n}$.

We define the function 
%
$\sigma: \mathbb{N} \rightarrow \mathbb{N}$
%
such that
%
$$\sigma(n) = \sum_{d|n}d$$
%
the sum taken over the divisors of $n$. We say $n$ is 
\textit{abundant} if $\sigma(n) \geq 2n$ \footnotemark
, \textit{perfect} if $\sigma(n) = 2n$, and \textit{deficient} if 
$\sigma(n) < 2n$. We say that $n$ is \textit{pseudoperfect} if 
$n$ is a non-perfect abundant and there exists a subset $ S 
\subset \textit{A}_{n}$ such that
%
\footnotetext{This is the definition Erd\H{o}s gave for abundant
numbers.\cite{erdos}}
%
%
$$ n = \sum_{d \in S} d .$$
%
We say that $n$ is \textit{weird} if $n$ is abundant but not 
pseudoperfect or perfect. The smallest weird number is 70, and 
all known weird numbers are even.

If $n$ is abundant and all of its proper divisors are deficient,
we say that $n$ is a primitive abundant number. The smallest primitive abundant number is 6, by our definition. Some authors exclude perfect
numbers from abundant numbers, in which case the smallest primitive abundant
number is 20.
This paper focuses
on odd primitive abundant numbers, which we refer to as OPANs
(Following the notation of \cite{valdas} and \cite{amato}). The sequence
of OPANs are available at \url{https://oeis.org/A006038}.\cite{oeis}

It is not known whether any odd perfect or odd weird numbers
exists. However, it is easy to show that any odd perfect number is
an OPAN, if such a number exists. Likewise if an odd weird number
$w$ exists, it can be shown that $w$ is an OPAN or $w$ is the
multiple of a weird OPAN.

Our purpose is to find an algorithm that enumerates all OPANs
with a fixed number of unique prime divisors $d$. With this, we
classify all OPANs with $5$ prime divisors as weird, perfect, or pseudoperfect.
We found that all OPANs with $3, 4$, or $5$ divisors are pseudoperfect,
which allows us to conclude that an odd weird number and an odd perfect number must have at least $6$ unique prime factors.

In this paper, we succeed in finding the described algorithm. 
We prove that the algorithm does indeed find all OPANs with $3, 4, 5, $ and $6$
unique prime divisors. We wish to generalize the proof to $d$ divisors.
Finding all OPANs with $6$ prime divisors 
was previously an unsolved problem\cite{amato}.

Through implementing the algorithm in Section 5, we were able to find the
number of OPANs with 6 divisors. That number is approximately $5.8
* 10^{13}$

The largest odd primitive abundant number with 6 divisors is:
$3^{38} 5^{28} 17^{16} 257^8 65537^4 4294967291^2$. It has 116 digits.



%Suppose we are able to find all OPANs with $d$ or fewer prime divisors, 
%and each of those OPANs are semiperfect. Then we know that an odd 
%weird number an odd perfect number must have at least $d + 1$ prime 
%divisors, if such numbers exist.

%Therefore a reasonable approach would be to find an algorithm to
%determine all odd primitive abundant numbers with a given number 
%of unique prime divisors, and then determine if any of those OPANs
%are weird or perfect. Such an algorithm will find one of the
%desired numbers, or impose a condition on the existance of odd
%weird and odd perfect numbers.

%If we can get this algorithm to run quickly enough, we will be
%able to determine the number of odd primitive abundant numbers
%with $6$ divisors, a currently unsolved problem\
\end{document}
