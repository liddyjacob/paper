\documentclass[../paper.tex]{article}


\begin{document}

\begin{abstract}
A primative abundant number is such that none of it's 
divisors (other than itself) are abundant numbers. Dickson proved
that there exists a finite number of primative odd abundant 
numbers with d prime divisors. In this paper we find an algorithm 
that finds all primative odd numbers with d divisors, and use that
algorithm to prove that an odd weird number must have at least six 
prime divisors.

%TODO: can we get above five divisors
\end{abstract}

\section{Introduction}
We define the function 
%
$\sigma(n):\mathbb{N} \rightarrow \mathbb{N}$
%
such that
%
$$\sigma(n) = \sum_{d|n}d$$
%
the sum taken over the divisors of $n$. We say $n$ is 
\textit{abundant} if $\sigma(n) \geq 2n$, \textit{perfect} if 
$\sigma(n) = 2n$, and \textit{deficient} if $\sigma(n) < 2n$ The 
proper divisors of $n$ are the positive divisors of $n$ that are
less than $n$. We denote this set by $\textit{A}_{n}$
%
We say that $n$ is \textit{pseudoperfect} if $n$ is abundant 
but not perfect and there exists a subset 
$ S \subset \textit{A}_{n}$ such that
%
$$ n = \sum_{d \in S} d$$
%
We say that $n$ is \textit{weird} if $n$ is abundant but not 
pseudoperfect or perfect. The smallest weird number is 70, and 
all known weird numbers are even.

\end{document}
