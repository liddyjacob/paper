\documentclass[../paper.tex]{article}


\begin{document}

\begin{abstract}
An abundant number is said to be primitive if none of its proper 
divisors are abundant.  Dickson proved that for an arbitrary
positive integer $d$ there exists only finitely many odd primative
abundant numbers having exactly $d$ prime divisors.
In this paper we describe algorithms
that find all primitive odd numbers with d divisors, and use that
algorithm to prove that odd weird and odd perfect numbers must
have at least six prime divisors.

%TODO: can we get above five divisors
\end{abstract}

\section{Introduction}
The proper divisors of a positive integer $n$ are the positive
divisors of $n$ that are less than $n$. We denote the set of 
proper divisors of n by $\textit{A}_{n}$.

We define the function 
%
$\sigma: \mathbb{N} \rightarrow \mathbb{N}$
%
such that
%
$$\sigma(n) = \sum_{d|n}d$$
%
the sum taken over the divisors of $n$. We say $n$ is 
\textit{abundant} if $\sigma(n) \geq 2n$ \footnotemark
,\textit{perfect} if $\sigma(n) = 2n$, and \textit{deficient} if 
$\sigma(n) < 2n$. We say that $n$ is \textit{pseudoperfect} if 
$n$ is a non-perfect abundant and there exists a subset $ S 
\subset \textit{A}_{n}$ such that
%
$$ n = \sum_{d \in S} d$$
%
We say that $n$ is \textit{weird} if $n$ is abundant but not 
pseudoperfect or perfect. The smallest weird number is 70, and 
all known weird numbers are even.
\footnotetext{This is the definition Erd\H{o}s gave for abundant
numbers.}

\end{document}
