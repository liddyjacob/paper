\documentclass[../paper.tex]{article}

\begin{document}

\section{Preliminaries}

We define $b(n): \mathbb{N} \mapsto \mathbb{Q}$ by
%
$$b(n) = \frac{\sigma(n)}{n} $$
%
Hence $n$ is abundant if and only if $b(n) \geq 2$

Note that if $n=p_1^{m_1}p_2^{m_2}p_3^{m_3}...p_{k}^{m_k}$,
It can be shown that 
%
$$\sigma(n) = \prod_{1 \leq i \leq k} \frac{p_i^{m_i + 1} -1}{p_i -1}$$
%
Thus, $b(n)$ can be expressed as:

$$b(n) = \frac{\alpha(n)}{n} = \prod_{1 \leq i \leq k} 
\frac{p_i^{m_i + 1} -1}{(p_i -1)p_{i}^{m_{i}}}$$

	Suppose $n=p_1^{m_1}p_2^{m_2}...p_{k}^{m_k}$, and suppose 
$n'=p_1^{m_1}p_2^{m_2}...p_i^{m_i + 1}...p_{k}^{m_k}$. We want
to define a function such that $b(n') = b(n) f$. We call this
function $\Delta_{+}(p_i, m_i)$, hence $b(n') = b(n) \Delta_{+}(p,m)$.
Thus 
%
$$\Delta_{+}(p_i, m_i) = \frac{b(n')}{b(n)} = 
\frac{p_i^{m_i+2} - 1}{ p_i (p_i^{m_i + 1} - 1)}$$
%

	Now suppose $n=p_1^{m_1}p_2^{m_2}...p_{k}^{m_k}$, and suppose 
$n_d=p_1^{m_1}p_2^{m_2}...p_i^{m_i - 1}...p_{k}^{m_k}$. We want
to define a function $g$ such that $b(n_d) = b(n) g$. We call this
function $\Delta_{-}(p_i, m_i)$, hence $b(n_d) = b(n) \Delta_{-}(p,m)$.
Thus 
%
$$\Delta_{-}(p_i, m_i) = \frac{b(n_d)}{b(n)} = 
\frac{p_i(p_i^{m_i} - 1)}{ p_i^{m_i + 1} - 1}$$
%

\begin{theorem} Consider the set of primes 
%
$P = \{p_{1}, p_{2}, p_{3}, .... , p_{k}\}$.
%
There exists infinity many abundant integer $n$ whose prime 
factors are in the set $P$  if and only if
%
$$\prod_{1 \leq i \leq k} (\frac{p_{i}}{p_{i} -1}) > 2$$
%
\end{theorem}

\begin{proof}

Note that
%  
$$\prod_{1 \leq i \leq k} (\frac{p_{i}}{p_{i} -1}) 
%
= \prod_{1 \leq i \leq k} \lim_{m_i \rightarrow \infty} 
%
( \frac{p_i^{m_{i} + 1} -1}{(p_i -1)p_{i}^{m_{i}}}). $$

First we assume that

  
$$\prod_{1 \leq i \leq k} \lim_{m_i \rightarrow \infty} 
( \frac{p_i^{m_{i} + 1} -1}{(p_i -1)p_{i}^{m_{i}}}) > 2$$

Hence there exists arbitrary integers $\{m_1, m_2, ... , m_k\}$ 
such that 
%
$$\prod_{1 \leq i \leq k} \frac{p_i^{m{i} + 1} -1}
{(p_i -1)p_{i}^{m_{i}}}) > 2$$
%

Thus we let $n=p_1^{m_1}p_2^{m_2}p_3^{m_3}...p_{k}^{m_k}$. 
Since $b(n) > 2$, it follows that $n$ is abundant. Note that an 
abundant number multiplied by a natural number is an abundant 
number. Hence $n$ can be multiplied by any one of its prime 
factors, and an abundant number will be the product. Thus there
are infinity many abundant integers $n$ whose prime factors are 
in the set $P$

Now we assume that there exists infinity many abundant integers 
$n$ whose prime factors are in the set $P$. Let $p_i \in P$. Note
that $np_i^r$ is abundant and has prime factors in $P$ for all 
natural numbers r. This implies that the exponents of the prime 
factors of $n$ can be increased and the product will remain abundant. Hence,
  $$\prod_{1 \leq i \leq k} \lim_{m_i \rightarrow \infty} ( \frac{p_i^{m_{i} + 1} -1}{(p_i -1)p_{i}^{m_{i}}}) > 2$$

\end{proof}

We call the limit product above $b_{\infty}(P)$ , given a set of primes
$P$. Suppose $P = \{p_1, p_2, ..., p_i \}$. We define $b_1(P)$ as
$b(p_1^1 p_2^1 ... p_i^1)$.

Another important function we will need describes what happens 
when some exponents are fixed and some exponents are allowed to
increase without limit. A mixed $b_{\infty}$ and $b$ function, 
say $mb$. The function will take a set of primes and exponents,
as well as the indicies of certain prime factors which are not
limited in exponent increase. Let $P$ be the sequence of primes 
and $E$ be the respective exponents for 
$p_1^{m_1} p_2^{m_2} ... p_d{m_d}$. Let $I$ be a set of indicies
ranging from $1$ to $d$, the indicies in which the corresponding
primes are desired to be raised without limit. Here is how $mb$
is defined:

$$mb(P,E,I) = \prod_{i \in I} b_{\infty}(p_i) 
\prod_{1 \leq j \leq d | j \notin I} b(p_j^{m_j})
$$

We now prove the following theorem, making use of the above definition:

\begin{theorem}
Let $p_1^{m_1} p_2^{m_2} ... p_d{m_d}$ be a prime factorization 
in which we choose to raise exponents of certain primes Let the 
indicies of these primes be the set $I$. Let I be non-empty. If
$mb(P,E,I) > 2$, then we can raise the exponents of primes with 
indicies in $I$ and eventually get an abundant number.
\end{theorem}

We (currently) leave the proof to the reader.

\end{document} 
