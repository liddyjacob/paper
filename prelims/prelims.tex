\documentclass[../paper.tex]{article}

\begin{document}

\section{Preliminaries}

We define $b(n): \mathbb{N} \mapsto \mathbb{Q}$ by
%
$$b(n) = \frac{\sigma(n)}{n} $$
%
Hence $n$ is abundant if and only if $b(n) \geq 2$

Note that if $n=p_1^{m_1}p_2^{m_2}p_3^{m_3}...p_{k}^{m_k}$ 
for distinct primes $p_1, ..., p_k$, it can be shown that 
%
$$\sigma(n) = \prod_{1 \leq i \leq k} \frac{p_i^{m_i + 1} -1}{p_i -1}$$
%
Thus, $b(n)$ can be expressed as:


\begin{equation}\label{b_stuff}
  b(n) = \frac{\sigma(n)}{n} = \prod_{1 \leq i \leq k} 
  \frac{p_i^{m_i + 1} -1}{(p_i -1)p_{i}^{m_{i}}}
  =\prod_{1 \leq i \leq k} b(p_i^{m_i})
\end{equation}

$b(n)$ also has the following properties. If $n$ and $m$ are positive
integers greater than 1, then 

\begin{equation}\label{inc_ineq}
  b(n) < b(nm) % \leq b(n)b(m)
\end{equation}

To show this, we first suppose that $m$ and $n$ share no prime factors.
Thus if the prime factorization of $n$ is $p_1^{s_1} p_2^{s_2} ... p_i^{s:_i}$ 
and the prime factorization of $m$ is $r_1^{l_1} r_2^{l_2} ... r_j^{l_j}$.
Then the unique prime factorization of $mn$ is 

$$mn = p_1^{s_1} p_2^{s_2} ... p_i^{s_i} r_1^{t_1} r_2^{t_2} ... r_j^{t_j}$$

Thus $b(mn) = b(n)b(m)$. Since $\sigma(m) \geq m + 1$(as $m$ and 
$1$ are divisors of $m$), we have that $b(m) \geq (m + 1) / m > 1$.
Thus,
$b(n) < b(n)b(m) = b(nm)$. 

%lmnopqrst

To show that that (\ref{inc_ineq}) is true in general,
we first note that given any prime $p$ and
any positive integer $m$, we can show that $b(p^m) < b(p^{m + 1})$.
Let $n = p_1^{s_1} p_2^{s_2}... p_d^{s_d}$ 
Let $m$ be written in the prime factorization 
$m = q_1^{t_1} q_2^{t_2} ... q_i^{t_i} 
r_1^{e_1} r_2^{e_2} ... r_j^{e_j}$ where 
$q_1, q_2, ..., q_i$ are prime factors that $n$ and $m$ share,
and each $r$ is a prime factor that divides $m$ but not $n$.
Suppose that $m$ and $n$ share at least one prime factor, as the 
other case was discussed in the previous paragraph.
%
Let $n' = n q_1^{s_1} q_2^{s_2} ... q_i^{s_i} =
p_1^{s'_1} p_2^{s'_2} ... p_d^{s'_d}$, where each $s'_i \geq s_i$,as
$n'$ has the same prime factors as $n$ but with some higer powers.
Hence $b(n) < b(n')$. Since $n'$ and $r_1^{l_1} r_2^{l_2} ... r_j^{l_j}$
share no prime factors, we have that 
$b(n') \leq b(n' r_1^{l_1} ... r_j^{l_j}) = b(nm)$.
Thus $b(n) < b(n') \leq b(mn)$. Hence, $b(n)$ is a 
multiplicativly increasing function.
\\

We want to see what happens to $b(n)$ when exponents in the prime
factorization of $n$ become larger and larger. Taking the limit of
$b(p^m)$ as $m$ goes to infinity, we have that:
$$\lim_{m \rightarrow \infty}(b(p^m)) = \frac{p}{p-1}$$
Applying this to a sequence of prime $P = \{p_1, p_2, ..., p_d\}$,
we denote the following

\begin{equation}\label{b_inf_def}
  b_{\infty}(P) 
  = \prod_{1 \leq i \leq k} \lim_{m_i \rightarrow \infty}(b(p^m))
                = \prod_{1 \leq i \leq k}(\frac{p_i}{p_i - 1})
\end{equation}


We call the limit product above $b_{\infty}(P)$ , given a set of primes
$P$. Suppose $P = \{p_1, p_2, ..., p_i \}$. We define $b_1(P)$ as
$b(p_1^1 p_2^1 ... p_i^1)$.


\begin{theorem}\label{b_inf_1}
Consider the set of primes 
%
$P = \{p_{1}, p_{2}, p_{3}, .... , p_{k}\}$.
%
There exists infinity many abundant integer $n$ whose prime 
factors compose the set $P$  if and only if $b_{\infty}(P) > 2$

\end{theorem}

\begin{proof}

First we assume that $b_\infty > 2$. Let $L = b_\infty(P)$. 
Note that
%  
$$L 
%
= \prod_{1 \leq i \leq k} \lim_{m_i \rightarrow \infty}(b(p_i^{m_i})) 
%
= \prod_{1 \leq i \leq k} \lim_{m_i \rightarrow \infty} 
%
( \frac{p_i^{m_{i} + 1} -1}{(p_i -1)p_{i}^{m_{i}}})$$

Hence , we can allow
$\prod_{1 \leq i \leq k} \frac{p_i^{m_{i} + 1} -1}
{(p_i -1)p_{i}^{m_{i}}}$ to be aribrarily close to L for large 
enough exponents $m$. Since L is greater than two, there exists a
collection of positive integers $m_1, m_2, ... , m_k$ such that 
% 
$$L > \prod_{1 \leq i \leq k} \frac{p_i^{m_{i} + 1} -1}
{(p_i -1)p_{i}^{m_{i}}} = \prod_{1 \leq i \leq k} b(p^{m_i}) > 2$$
%

Thus we let $n=p_1^{m_1}p_2^{m_2}p_3^{m_3}...p_{k}^{m_k}$. 
Since $b(n) > 2$, it follows that $n$ is abundant. 
Because $b(n)$ is an increasing function, $n$ can be multiplied
by any one of its prime factors, and an abundant number 
will be the product. In other words,
%
$$2 < b(n) < b(np_1) < b(np_1^2) < ...$$
%
Thus there are infinity many abundant integers $n$ whose prime
factors are in the set $P$.

Now we assume that there exists infinitly many abundant integers 
$n$ whose prime factors compose $P$. Let $p_i \in P$. Given any
positive integer $r$, we have that that $np_i^r$ is abundant. 
Hence,
%
$$\prod_{1 \leq i \leq k} \lim_{m_i \rightarrow \infty} 
( \frac{p_i^{m_{i} + 1} -1}{(p_i -1)p_{i}^{m_{i}}}) > 2$$

\end{proof}

\begin{theorem}
If $b_{\infty}(P) \leq 2$, then there are no abundant numbers 
whose prime factors compose $P$.
\end{theorem}

\begin{proof}
Let $P = \{p_1, p_2, ..., p_d\}$
Since $b_{\infty}(P) \leq 2$, and $b(p_1^{m_1} p_2^{m_2} ... p_d^{m_d})$
is increasing as the exponents $m_1, m_2, ..., m_d$ increase, we have that
$b(p_1^{m_1} p_2^{m_2} ... p_d^{m_d}) < b_{\infty}(P) \leq 2$ for any 
given exponents. Thus $b(n) < 2$ for any $n$ whose prime factors
compose $P$. Thus there does not exist any abundant number whose 
prime factors compose $P$.

\end{proof}

Another important function we will need describes what happens 
when some exponents are fixed and some exponents are allowed to
increase without limit. A mixed $b_{\infty}$ and $b$ function, 
say $mb$. The function will take a set of primes and exponents,
as well as the indicies of certain prime factors which are not
limited in exponent increase. Let $P$ be the sequence of primes 
and $E$ be the respective exponents for 
$p_1^{m_1} p_2^{m_2} ... p_d^{m_d}$. Let $I$ be a set of indicies
ranging from $1$ to $d$, the indicies in which the corresponding
primes are desired to be raised without limit. We define $mb$ as 
follows:

\begin{equation}\label{mb_def}
  mb(P,E,I) = \prod_{i \in I} b_{\infty}(p_i) 
  \prod_{1 \leq j \leq d | j \notin I} b(p_j^{m_j})
\end{equation}

We now prove the following theorem, making use of the above definition:

\begin{theorem}
Let $p_1^{m_1} p_2^{m_2} ... p_d^{m_d}$ be a prime factorization 
in which we choose to raise exponents of certain primes Let the 
indicies of these primes be the set $I$. Let I be non-empty.
$mb(P,E,I) > 2$ if and only if we can raise the exponents of primes with 
indicies in $I$ and eventually get an abundant number.
\end{theorem}

We (currently) leave the proof to the reader. It is extremely 
similar to the proof of Theorem {\ref{b_inf_1}}.

Define the multivariable function $\Delta_+(p,m)$ as
$$\Delta_+(p,m) = \frac{p^{m+2} - 1}{ p (p^{m + 1} -
1)}$$. 
Also, define $\Delta_-(p,m)$ as 
$$\Delta_-(p,m) = \frac{p(p^{m} - 1)}{ p^{m + 1} - 1}$$.

These functions have the following useful properties:

\begin{theorem}[Property]\label{dp_def}
Suppose $n = p_1^{m_1} p_2^{m_2} ... p_i^{m_i} ... p_d^{m_d}$.
Then,

$$ b(n) * \Delta_+(p_i, m_i) = b(p_i * n)$$

\end{theorem}

\begin{theorem}[Property]\label{dm_def}
Suppose $n = p_1^{m_1} p_2^{m_2} ... p_i^{m_i} ... p_d^{m_d}$ and
that $n' = p_1^{m_1} p_2^{m_2} ... p_i^{m_i - 1} ... p_d^{m_d}$. Then,

$$ b(n) * \Delta_-(p_i, m_i) = b(n')$$

\end{theorem}

The proofs for {\ref{dp_def}} and {\ref{dm_def}} are nearly identical.
In fact, $\Delta_+$ and $\Delta_-$ were constructed such that
these properties would be true. Using a bit of algebra, it is not
hard to show that these propeties are true.

\end{document} 
