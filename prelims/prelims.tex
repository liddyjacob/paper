\documentclass[../paper.tex]{article}

\begin{document}

\section{Preliminaries}

We define $b(n): \mathbb{N} \mapsto \mathbb{Q}$ by
%
$$b(n) = \frac{\sigma(n)}{n} $$
%
Hence $n$ is abundant if and only if $b(n) \geq 2$

Note that if $n=p_1^{m_1}p_2^{m_2}p_3^{m_3}...p_{k}^{m_k}$,
It can be shown that 
%
$$\sigma(n) = \prod_{1 \leq i \leq k} \frac{p_i^{m_i + 1} -1}{p_i -1}$$
%
Thus, $b(n)$ can be expressed as:

$$b(n) = \frac{\alpha(n)}{n} = \prod_{1 \leq i \leq k} 
\frac{p_i^{m_i + 1} -1}{(p_i -1)p_{i}^{m_{i}}}$$

	Suppose $n=p_1^{m_1}p_2^{m_2}...p_{k}^{m_k}$, and suppose 
$n'=p_1^{m_1}p_2^{m_2}...p_i^{m_i + 1}...p_{k}^{m_k}$. We want
to define a function such that $b(n') = b(n) f$. We call this
function $\Delta_{+}(p_i, m_i)$, hence $b(n') = b(n) \Delta_{+}(p,m)$.
Thus 
%
$$\Delta_{+}(p_i, m_i) = \frac{b(n')}{b(n)} = 
\frac{p_i^{m_i+2} - 1}{ p_i (p_i^{m_i + 1} - 1)}$$
%



\end{document} 
