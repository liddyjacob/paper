\documentclass[../paper.tex]{subfiles}

\begin{document}

\section{Prime Sequences: Theorems and Definitions}

\begin{theorem} \label{Primechops}
Suppose that $b_{\infty}(\{p_1,p_2, ..., p_i\}) > 2$,
where $p_1 < p_2 < p_3 < ... < p_i$. Then there exists some
positive integer $\ell$ less than $i$ such that $\{p_1, p_2, p_3, 
..., p_\ell\}$ compose the prime factors of some primitive 
abundant odd.
\end{theorem}

\begin{proof}
Let $\ell$ be the smallest positive integer integer such that 
$b_{\infty}(\{p_1, p_2, ..., p_\ell\}) > 2$. Clearly $l > 2$ as 
$2 > b_{\infty}(\{2\}) \geq b_{\infty}(\{p\})$ for any prime $p$.
Then we have that 
$b_{\infty}(p_1 p_2 ... p_{\ell - 1}) \leq 2$. Since 
$b_{\infty}(p_{\ell - 1}) \leq b_{\infty}(p_{k})$ for any
positive integer $k$ such that  $k \leq \ell - 1$, we have that
removing any prime from the sequence $P = \{p_1, p_2, ..., p_{\ell} \}$ will
cause $b_{\infty}(P)$ to be less than two. By Theorem
{\ref{b_inf_n!}} there does not exist
an abundant odd whose prime divisors are a proper subsequence of
$P$.
\end{proof}

\begin{theorem} Suppose there exists an odd weird number. Let $w$
denote some odd weird number with the smallest possible number of
distinct prime divisors. Then $w$ is primitive abundant.
\end{theorem}

\begin{coro} If no primitive weird abundant odd exists
having fewer than $d$ distinct prime divisors, then no odd weird
exists having fewer than $d$ distinct prime divisors divisors.
\end{coro}


\begin{theorem}[Continuity of Primitive Abundant
Numbers]\label{Continuity}
Let $P = \{p_0, p_1, ..., p_{d-2}, r\}$ be an increasing sequence 
of primes. Suppose that $n = \nu(P, M)$ is a primitive abundant 
number for some sequence of positive integers $M$. Let $s$ be a 
prime such that $p_{d-2} < s < r$, and let $P' = \{p_0, p_1,...,
p_{d-2}, s\}$. Then there exists some sequence of positive integers
$E$ such that $\nu(P', E)$ is a primitive abundant number.
\end{theorem}

\begin{proof}
Suppose that $n = \nu(P,M)$ is a primitive abundant number for
some sequence of positive integers $M$. We write $M = \{ m_0,
m_1, ..., m_{d-1}\}$. Then $n = p_0^{m_0} p_1^{m_1} ... 
p_{d-2}^{m_{d-2}} r^{m_{d-1}}$. We want to consider the integer 
$\ell = \nu(P', M)$. It is easy to show that $\ell$ is abundant. 
If $\ell$ is primitive, then we are done. Suppose that $\ell$ is not
primitive. We know that $\ell$ is a multiple of some primtive
abundant number $\ell'$. We will argue that all prime factors in
$P'$ must divide $\ell'$. We will use proof by contradiction. 

Let $h = p_0^{m_0} p_1^{m_1} ... p_{d-2}^{m_{d-2}}$. Note that $h$
is a proper divisor of $n$, and therefore $h$ is deficient.

Suppose, to the contrary, that $p_i$ does not divide $\ell'$ for
some $i$ such that $0 \leq i \leq d - 2$
Then $\ell'$ is a multiple of $k = p_0^{m_0}
p_1^{m_1} ... p_{i-1}^{m_{i-1}} p_{i+1}^{m_{i+1}} ... 
s^{m_{d-1}}$. Notice that $b(k) < b(h) < 2$. However, k is a 
multiple of the abundant number $\ell'$, a contradiction. We have
shown that all primes in the set $\{p_0, p_1, ..., p_{d-2}\}$
must divide $\ell'$.  

Now we suppose that $s$ does not divide $\ell'$. This implies
that $h$ is a multiple of $\ell'$. However, $\ell'$ is abundant 
and therefore $h$ is abundant. This is a contradiction.

Thus $\ell'$ is a primitive abundant number whose prime divisors
compose $P'$.
\end{proof}


Let $P$ be an
ordered sequence of $i$ primes. Suppose there exists at least one
odd primitive abundant number $n$ having $k$ divisors, such that $i
< k$ and $P$ composes the smallest $i$ prime divisors of $n$.
Such a number $n$ is referred to as a "$P$-initiated OPN"

\begin{theorem}[Prime Divisibility of Abundant Numbers]
\label{Divisibility}
Let $P$ be an increasing sequence of primes. Let $r$ and $s$ be
primes such that $max(P) < r < s$. Suppose there exists a $(P +
s)$-initiated OPN with $d$ divisors. Then there exists a $(P +
r)$-initiated OPN with at most $d$ divisors. 
\end{theorem}

We will prove the equivalent statement \textit{"If
there does not exist a $(P + r)$-initiated OPN with $d$ divisors,
then there must exists a $(P + r)$-initiated OPN with less than
$d$ divisors.}

\begin{proof}
By hypothesis, there exists a $P + s$-initiated OPN with $d$ divisors.
Suppose $n$ is such an $OPN$. We know that $n = \nu(P + s, M_p) 
\nu(Q, M_q)$ for sequences of positive integers $M_p, M_q$, and 
some increasing sequence of primes $Q$ such that each element of $Q$ is 
greater than $r$. 

  We want to consider $m = \nu(P + r, M_p) \nu(Q, M_q)$. This 
integer is abundant as $s < r$. However, it is not primitive by 
our assumption.


Consider the maximum integer $k$ such that $r^k$ divides $n$.
Note that $\ell = n / r^{k}$ is a proper divisor of $n$ and
thus is deficient. Notice that $\ell$ is also a divisor of $m$, 
and thus a primitive abundant divisor of $m$ must not be a 
divisor of $\ell$. Thus a primitive abundant divisor of $m$ must 
be divisible by $s$.

  It is easy to show that a primitive abundant divisor of $m$ must
also be divisible by each of $p \in P$  using Theorem
{\ref{prime_ineq}}. After showing this, we can see that the 
there exists $P + s$-initiated OPN with less than $d$ divisors.
\end{proof}

We now define a certain class of primes.
Suppose $P$ is an ordered sequence of $i$ primes. Suppose $d$ is
an integer greater than $i$. Let $p$ be the largest prime such 
that there exists a $(P + p)$-initiated OPA having less than $d$ 
prime divisors\footnotemark. 
% -------
\footnotetext{the notion $P + p$ intends that $p$ is appended to the end
of the sequence $P$}
% -------
We know this prime $p$ exists since 
there exist only finitly many primitive abundant numbers having
fewer than $d$ prime divisors(Dickson). We denote such a prime $p$
as $Cap_d(P)$. 

\begin{theorem}
Let $P$ be an ordered sequence of primes, and  let $p$ and $q$ be 
primes such that $max(P) < r < s$. Suppose $d$ is an integer such
that $d > |P|$. Finally, suppose that $r > Cap_d(P)$. If there
does not exist any $(P + r)$-initiated OPN's with $d$ divisors,
then there does not exist any $(P + s)$-initiated OPN's with $d$ 
divisors.
\end{theorem}

\begin{proof}
Assume, to the contrary, that there does exist a $(P + 
s)$-initiated OPN with $d$ divisors. By Theorem {\ref{Divisibility}},
there exists a $(P + r)$-initated OPN having fewer than $d$
divisors. However, $r > Cap_d(P)$, and by definition of
$Cap_d(P)$, there cannot exist a $(P + r)$-initiated OPN having
fewer than $d$ divisors. This is a contradiction.
\end{proof}

\end{document}
