\documentclass[../paper.tex]{subfiles}


\begin{document}

\section{Preliminary Theorems}

%TODO: Mention that p_1, p_2, ..., p_d are always listed in increasing 
% order unless said not to be.


% 
% ---------- RELEVANCE OF THE BELOW CONJECTIURE ----------
%		This conjecture allows us to conclude that if some set
% of primes p_1, p_2, ..., p_{i-1}, q, is NOT going to be
% the first i primes of some prim. ab. odd with d divisors
% then NO prime after q will. For if not, this conjecture 
% woudl be false. There would be a 'hole' in the primes
% and in the graph. 
% --------------------------------------------------------
% NOTE: Need to prove that for some primes p_1 p_2 .. p_i
% if these are the start of some prim ab odd with d divisrs
% then exists consecutive primes q_i+1, q_i+2, ..., q_d s.t.
% there exists prim ab odd with prime divisors p-1 p_2 ...
% p_i q_i+1 q_i+2 ... q_d.

% This will allow us to 'construct' consecutive primes for 
% r.

% Repeat of NOTE  below, reworded.

% So yet another theorem will be needed: Given that there 
% exists a seq of primes P that represent the first i primes
% for some prim ab odd, then there exists a prim ab odd
% whose first primes are P and all after are consecutive 
% starting at q_1. 

% So we can consider the consecutive set of primes following
% r. This conscutive set will be either the primes following
% r itself or will be the smallest consecutive set following
% q_1. This should avoid r not being non-primative

% This will allow us to find the primes after r.

% Possibly important: For the max q such that p_1, p_2, ...
% p_i-1, q,  such that these are the first i divisors of 
% some prim ab odd with d > i divisors, then that prim
% ab odd will have consecutive primes after q.
\textbf{Conjecture?:} Suppose that there exists a primative abundant
odd whose first i prime divisors are $p_1, p_2, p_3, ..., p_{i-1}, 
q_1$ and a primative abundant odd whose prime divisors are $p_1, p_2, 
..., p_{d-1}, q_2$ with $q_2 \geq q_1$. Then for each prime $r$ 
between $q_1$ and $q_2$ there exists a primative abundant odd whose 
first $i$ prime divisors are $p_1, p_2, p_3, ..., p_{i-1}, r$.

\textit{Proof:} 
%
Since there exists some primative abundant whose first d divisors
are ... there exists a primative abundant such that 
.
.
.


\textbf{Corollary:} 
Let $n \leq d$. Let $P = \{p_1, p_2, ..., p_n\}$ be a set of primes
such that $p_1 p_2 p_3 ... p_n$ is not an abundant number.
If there does not exist a primative abundand odd with $d$ divisors
whose first $n$ prime divisors compose P, then if $q$ is a prime 
greater than $p_d$, there does not exist a primative abundant odd
with $d$ divisors whose first $n$ divisors are $p_1, p_2, ...,
p_{n-1}, q$.

\textbf{Theorem:} If there exists a weird number that is odd, 
then the smallest odd weird number will be primative abundant.

\textbf{Corollary:} If no primative weird abundant odd exists
with d or less divisors, then no odd weird exists with d or less
divisors.

\end{document}
