\documentclass[../paper.tex]{subfiles}

\begin{document}

\section{Preliminary Theorems}

\section*{Lemmas And Theorems}\par\hspace{4ex}
\textbf{Lemma 1: } Consider the set of primes 
%
$P = \{p_{1}, p_{2}, p_{3}, .... , p_{k}\}$.
%
There exists infinity many abundant integer $n$ whose prime 
factors are in the set $P$  if and only if
%
$$\prod_{1 \leq i \leq k} (\frac{p_{i}}{p_{i} -1}) > 2$$
%

\begin{proof}

Note that
%  
$$\prod_{1 \leq i \leq k} (\frac{p_{i}}{p_{i} -1}) 
%
= \prod_{1 \leq i \leq k} \lim_{m_i \rightarrow \infty} 
%
( \frac{p_i^{m_{i} + 1} -1}{(p_i -1)p_{i}^{m_{i}}}). $$

First we assume that

  
$$\prod_{1 \leq i \leq k} \lim_{m_i \rightarrow \infty} 
( \frac{p_i^{m_{i} + 1} -1}{(p_i -1)p_{i}^{m_{i}}}) > 2$$

Hence there exists arbitrary integers $\{m_1, m_2, ... , m_k\}$ 
such that 
%
$$\prod_{1 \leq i \leq k} \frac{p_i^{m{i} + 1} -1}
{(p_i -1)p_{i}^{m_{i}}}) > 2$$
%

Thus we let $n=p_1^{m_1}p_2^{m_2}p_3^{m_3}...p_{k}^{m_k}$. 
Since $b(n) > 2$, it follows that $n$ is abundant. Note that an 
abundant number multiplied by a natural number is an abundant 
number. Hence $n$ can be multiplied by any one of its prime 
factors, and an abundant number will be the product. Thus there
are infinity many abundant integers $n$ whose prime factors are 
in the set $P$

Now we assume that there exists infinity many abundant integers 
$n$ whose prime factors are in the set $P$. Let $p_i \in P$. Note
that $np_i^r$ is abundant and has prime factors in $P$ for all 
natural numbers r. This implies that the exponents of the prime 
factors of $n$ can be increased and the product will remain abundant. Hence,
  $$\prod_{1 \leq i \leq k} \lim_{m_i \rightarrow \infty} ( \frac{p_i^{m_{i} + 1} -1}{(p_i -1)p_{i}^{m_{i}}}) > 2$$

\end{proof}

We call the limit product above $b_{\infty}(P)$ , given a set of primes
$P$. Suppose $P = \{p_1, p_2, ..., p_i \}$. We define $b_1(P)$ as
$b(p_1^1 p_2^1 ... p_i^1)$. \\

\textbf{FALSE Lemma:} Let $P$ be a set of $d$ odd primes. If $b_{1}(P) \leq 2$
and $b_{\infty}(P) > 2$, then $P$ is the set of prime divisors of 
some primative abundant odd with $d$ divisors.\\ COUNTEREXAMPLE:
3 5 7 1003 not P. AB. ever.

\begin{proof}

Let $M$ be a set of $d$ positive integers that will represent
the exponents of a primative abundant number whose prime factors
compose $P$. We 'initialize' M to be
%
$M = \{m_1=1, m_2=1, ..., m_d=1 \}$. 
%
First we check if raising any exponent $m_i$ by $1$ will make 
$p_1^{m_1} p_2^{m_2} ... p_d^{m_d}$ abundant. We will break off
by cases depending on if this is possible.
\textbf{case 1}
	Suppose raising some exponent by 1 will make $n = p_1^{m_1} 
p_2^{m_2} ... p_d^{m_d}$ abundant. Find all exponents that can be
raised to make the product abundant, and select the exponent $m_i$
such that $\Delta_{+}(p_i, m_i)$ is smallest. Raise this exponent,
and call the resultant product $n'$
%Or pick largest Delta_-??

When this exponent is raised, we will have an abundant number. It
remains to prove that this number will also be primative. Since
we have that $b(n') = \Delta_{+}(p_i, m_i) b(n) \geq 2$. Since 
this is the minimum $\Delta_{+}$, we have that any other ...

\textbf{case 2}
If there is one 
%


\end{proof}


\textbf{Theorem:} Let $P$ be a set of $d$ odd primes. If $b_{1}(P) \leq 2$
and $b_{\infty}(P) > 2$, then the following algorithm will extract all primative
abundant odd numbers whose prime divisors compose $P$.

% The algorithm:
%
% * Set initial exponents to 1 
%
% * Check if abundant. If so check if primative
%   * If not primative, no abundant odds 
%		* If primative, this is 
%
% * If b(p1^e1 p2^e2 ... pd^ed) >= 2
%		* If primative this is an option : return true
%		* If not return false
%
%	* Check to see if any exponent increases will cause abundance
%    If any do cause abundance, check if primative.
%    Always start with the exponents of highest 'influence'
%		 then move on to lower 'influence' exponents. influence
%    is measured by which exponents increase b(n) the most
%    Del_pos is exactly the measure of this.
%
%	* Once the first exponent which does not cause abundance is found
%    we want to check which exponents can be raised to 'infinity'
%    to eventually produce an abundant number. We raise the exponents
%    NEED TO TALK ABOUT ORDER OF INFLUENCE HERE. IN THE ALGORITHM
%    WE ASSUME INFINITE INFLUENCE IS DIRECTLY RELATED TO FINITE
%    SINGLE EXPONENT INFLUENCE? IS THIS TRUE? I DONT KNOW.
%				*Infinite influence: Take a prime-exp pair p^e -> p^infinity
%				*Finite influence  : Take a prime-exp pair p^e -> p^(e+1)
%       Are they related?
% ^ * Mixed b: Raise exponents of low influence pairs to see if they
%     mixed with others will eventually go on to infinity
% ^ * If so raise 

% THIS is going to be UGLY.
% NEW THEorem added - may be important



\textbf{Theorem:} Suppose that $b_{\infty}(p_1 p_2 ... p_i) > 2$,
where $p_1 < p_2 < p_3 < ... < p_i$. Then there exists some 
$\ell \leq i$ such that $p_1, p_2, p_3, ..., p_\ell$ are the prime
factors to some primative abundant odd.

\begin{proof}
	Consider the first $\ell$ such that $b_{\infty}(
p_1 p_2 ... p_\ell) > 2$. Then we have that $b_{\infty}(p_1 p_2 
... p_{\ell - 1}) \leq 2$. Since 
$b_{\infty}(p_\ell - 1) < b_{\infty}(p_{\ell - k})$, we have that
removing any prime from the set $P = \{p_1, p_2, ..., p_{\ell} \}$ will
cause $b_{\infty}$ to be less than two. Thus there does not exist
a abundant odd whose prime divisors are all in the set 
$P$ but do not cover all elements of $P$. Hence if an odd abundant
has prime divisors who compose $P$, then it is either a primative 
abundant odd or a multiple of a primative abundant odd whose prime
divisors compose $P$. Thus there exists a primative abundant odd
whose prime divisors compose $P$.
\end{proof}


\textbf{Lemma:} Suppose that $P = \{ p_1 \geq 3, p_2, ..., p_i\}$ are 
the first $i$ divisors of some primative abundant odd which has
$d$ prime divisors. Then there exists consecutive prime divisors
$Q = q_1, q_2, ..., q_{d - 1}$ with $p_i < q_1$ such that 
$P \cup Q$ compose the prime divisors of some primative abundant 
odd.

\begin{proof}
	Since $P$ consists of the first $i$ prime divisors of some 
primative abundant n, there exists at least one
$R = {r_1 > p_1, r_2, ..., r_{d-i}}$ such that $P \cup R$ are the 
prime divisors of some primative abundant odd. Choose an R with the
largest $r_1$ possible. Since there is a finite number of 
primative abundant odds with $d$ divisors we can choose such an
$r_1$. We wish to argue that this $r_1$ is in fact a possibility
for $q_1$.

	Suppose that the $R$ chosen was a set of sequential primes. 
Then we are done and the Theorem is true.

	Suppose that $R$ is non-sequential. We want to show that the 
sequence of primes following $r_1$ is the desired sequence of 
primes to satisfy the theorem. Call this sequence $R'$. Note that
$b_{\infty}(P \cup R) = b_{\infty}(P) b_{\infty}(R) > 2$. Given 
any $j \in \{1, 2, ..., d-i\}$ we have that $r'_j \geq r_j$, since
$R'$ is sequential. Thus $b_{\infty}(R') \geq b_{\infty}(R)$. Hence
we have that $b_{\infty}(P \cup R') > 2$. Thus for high enough 
exponents, 
%
$$p_1^{m_1}  p_2^{m_2} ... p_i^{m_i} (r'_1)^{m_{i+1}} ... (r'_{d-i})^{m_d}$$
%
is an abundant odd.

	It remains to prove that we can find a primative abundant odd 
of the above form.

	Note that there exists exponents such that 
%
$p_1^{e_1}  p_2^{e_2} ... p_i^{e_i} r_1^{e_{i+1}} ... r_{d-i}^{e_d}$
%
is primative abundant odd (Here it is not r' but r). 
Suppose, to the contrary, that there 
does not exists exponents to make $P \cup R'$ the prime divisors 
of some primative abundant odd. Thus any abundant odd whose prime
factors are in $P \cup R'$ are the product of some primative 
abundant odd whose prime factors are all in $P \cup R'$ but lacks
a single divisor in $P \cup R'$. If this is the case, is $r_1$ truely
the largest prime such that $P \cup R$?

Choose exponents $e_1, e_2, e_3, ..., e_d$. Now consider the following
$$p_1^{e_1} p_2^{e_2} ... p_i^{e_i} (r'_1)^{e_{i+1}} ... (r'_{d-i})^{e_d}$$
We want to prove that this is a multiple of a primative abundant 
number who shares exactly the same distinct prime factors. We know 
that $p_1^{e_1} p_2^{e_2} ... p_i^{e_i} (r'_1)^{e_{i+1}}$ is 
deficient since it is a divisor of a primative abundant odd.
(Recall that $r'_1 = r_1$). Find the first $r_k$ with $k <= d - i$
where 
$$p_1^{e_1} p_2^{e_2} ... p_i^{e_i} (r'_1)^{e_{i+1}} (r'_2)^e_{i+2} ... 
(r'_k)^e_{i+k})$$
is abundant. Note that these primes are primative abundant. 

\textbf{TODO: Finish ME!!!!!!!!!!!!!!!!!!!?}
%TODO:

\end{proof}


 

%TODO: Mention that p_1, p_2, ..., p_d are always listed in increasing 
% order unless said not to be.


% 
% ---------- RELEVANCE OF THE BELOW CONJECTIURE ----------
%		This conjecture allows us to conclude that if some set
% of primes p_1, p_2, ..., p_{i-1}, q, is NOT going to be
% the first i primes of some prim. ab. odd with d divisors
% then NO prime after q will. For if not, this conjecture 
% woudl be false. There would be a 'hole' in the primes
% and in the graph. 
% --------------------------------------------------------
% NOTE: Need to prove that for some primes p_1 p_2 .. p_i
% if these are the start of some prim ab odd with d divisrs
% then exists consecutive primes q_i+1, q_i+2, ..., q_d s.t.
% there exists prim ab odd with prime divisors p-1 p_2 ...
% p_i q_i+1 q_i+2 ... q_d.

% This will allow us to 'construct' consecutive primes for 
% r.

% Repeat of NOTE  below, reworded.

% So yet another theorem will be needed: Given that there 
% exists a seq of primes P that represent the first i primes
% for some prim ab odd, then there exists a prim ab odd
% whose first primes are P and all after are consecutive 
% starting at q_1. 

% So we can consider the consecutive set of primes following
% r. This conscutive set will be either the primes following
% r itself or will be the smallest consecutive set following
% q_1. This should avoid r not being non-primative

% This will allow us to find the primes after r.

% Possibly important: For the max q such that p_1, p_2, ...
% p_i-1, q,  such that these are the first i divisors of 
% some prim ab odd with d > i divisors, then that prim
% ab odd will have consecutive primes after q.

\textbf{Conjecture?:} Suppose that there exists a primative abundant
odd whose first i prime divisors are $p_1, p_2, p_3, ..., p_{i-1}, 
\ell_i$ and a primative abundant odd whose prime divisors are $p_1, p_2, 
..., p_{d-1}, u_i$ with $\ell_i \geq u_i$. Then for each prime $r+i$ 
between $q_1$ and $q_2$ there exists a primative abundant odd whose 
first $i$ prime divisors are $p_1, p_2, p_3, ..., p_{i-1}, r$.

\textit{Proof:} 
%
Since there exists some primative abundant whose first d divisors
are $p_1, p_2, ..., p_{d-1}, u_i$ we have that there exists a primative
abundant odd whose first 
.
.
.


\textbf{Corollary:} 
Let $n \leq d$. Let $P = \{p_1, p_2, ..., p_n\}$ be a set of primes
such that $p_1 p_2 p_3 ... p_n$ is not an abundant number.
If there does not exist a primative abundand odd with $d$ divisors
whose first $n$ prime divisors compose P, then if $q$ is a prime 
greater than $p_d$, there does not exist a primative abundant odd
with $d$ divisors whose first $n$ divisors are $p_1, p_2, ...,
p_{n-1}, q$.

\textbf{Theorem:} If there exists a weird number that is odd, 
then the smallest odd weird number will be primative abundant.
In fact, all weird numbers will have at least as many prime
divisors as the smallest weird odd number.

\textbf{Corollary:} If no primative weird abundant odd exists
with d or less divisors, then no odd weird exists with d or less
divisors.

\end{document}
