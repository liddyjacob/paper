\documentclass[../paper.tex]{subfiles}

\begin{document}

\section{Prime Sequences: Theorems and Definitions}

We start this section with some useful definitions to simplify
notation for the theroems that follow.

Suppose that $P = \{p_0, p_1, ..., p_d\}$ is an increasing 
sequence of prime numbers. Suppose $q$ is a prime number greater
than $max(P)$. We define the sequence of increasing primes $P + q$
by $P + p = \{p_0, p_1, ..., p_d, q\}$.

Let $P$ be an
ordered sequence of $i$ primes. Suppose there exists at least one
odd primitive abundant number $n$ having exactly $k$ prime divisors, 
such that $i < k$ and $P$ composes the smallest $i$ prime divisors of $n$.
Such a number $n$ is referred to as a "$P$-initiated OPN".

\begin{theorem} \label{Primechops}
Suppose that $b_{\infty}(\{p_1,p_2, ..., p_i\}) > 2$,
where $p_1 < p_2 < p_3 < ... < p_i$. Then there exists some
positive integer $\ell$ less than or equal to $i$ such that 
$\{p_1, p_2, p_3, ..., p_\ell\}$ compose the prime factors of
some primitive abundant odd.
\end{theorem}

\begin{proof}
Let $\ell$ be the smallest positive integer such that 
$b_{\infty}(\{p_1, p_2, ..., p_\ell\}) > 2$. Note that
$2 \geq b_{\infty}(\{2\}) \geq b_{\infty}(\{p\})$ for any prime $p$.
Hence by Theorem {\ref{b_inf_n!}}, we have that there does not exists a
primtive abundant odd with one prime divisor. Thus $\ell \geq 2$.

For any positive integer $k$ such that $k < \ell - 1$, we have
that $b_{\infty}(p_k) > b_{\infty}(p_l)$. Thus for any proper 
subsequence $S$ of $\{p_0, p_1, ..., p_{\ell}\}$, we have that 
$b_{\infty}(S) \leq 2$. Thus there exists a primitive abundant
number whose divisors compose $\{p_0, p_1, ..., p_{\ell}$.
\end{proof}

\begin{theorem} Suppose there exists an odd weird number. Let $w$
denote some odd weird number with the smallest possible number of
distinct prime divisors. Then $w$ is primitive abundant.
\end{theorem}

\begin{coro} If no primitive weird abundant odd exists
having fewer than $d$ distinct prime divisors, then no odd weird
exists having fewer than $d$ distinct prime divisors.
\end{coro}


\begin{theorem}[Prime Divisibility of Abundant Numbers]
\label{Divisibility}
Let $P$ be an increasing sequence of odd primes. Let $r$ and $s$ be
primes such that $max(P) < r < s$. Suppose there exists a $(P +
s)$-initiated OPN with $d$ divisors. Then there exists a $(P +
r)$-initiated OPN with at most $d$ divisors. 
\end{theorem}

\begin{proof}
By hypothesis, there exists a $(P + s)$-initiated OPN with $d$ divisors.
Suppose $n$ is such an $OPN$. Note that $n$ is in the form $n =
\nu(P + s, M_{(P + s)}) \nu(Q, M_Q)$ for sequences of positive 
integers $M_p, M_q$, and some increasing sequence of primes $Q$.
Also, $s < min(Q)$.

  We want to consider $m = \nu(P + r, M_{(P + s)}) \nu(Q, M_Q)$. 
This integer is abundant as $s < r$. If this integer is primitive,
we are done. If not, then we will show that any primitive abundant
divisor of $m$ must be $(P + r)$-initiated(Proving the theorem).

  Suppose that $m$ is not primitive. Consider the maximum integer
$k$ such that $r^k | m$. Notice that $d = m / r^k$ is a divisor of $n$
as $m / r^k = n / s^k$. Thus $d$ is deficient and any divisor of
$d$ is therefore deficient. Thus an abundant divisor of $m$ must
be divisible by $r$.

Let $p \in P$.  We also can show that $2 > b(m / r^k) > b(m / p^t)$, 
where $t$ is the maximum integer such that $p_t | m$: Note that
$b(m / r^k) = b(m) / b(r^k)$, and likewise $b(m / p^t) = b(m) /
b(p^t)$. Also, $b(r^k) < b(p)$ by Theorem {\ref{prime_ineq}}. Thus
$2 > b(m / r^k) > b(m / p^t)$. Thus any abundant divisor of $m$
must be divisible by each prime in $P + r$. Therefore any
primitive divisor of $m$ must be divisible by $P + r$. As
there must exists a primitive abundant divisor of $m$, there must
exist a $(P + r)$-initiated OPN with at most $d$ divisors.
\end{proof}

\begin{coro}[Continuity of Primitive Abundant
Numbers]\label{Continuity}
Let $P = \{p_0, p_1, ..., p_{d-2}, r\}$ be an increasing sequence 
of primes. Suppose that $n = \nu(P, M)$ is a primitive abundant 
number for some sequence of positive integers $M$. Let $s$ be a 
prime such that $p_{d-2} < s < r$, and let $P' = \{p_0, p_1,...,
p_{d-2}, s\}$. Then there exists some sequence of positive integers
$E$ such that $\nu(P', E)$ is a primitive abundant number.
\end{theorem}

\begin{proof}
This is a special case of the previous theorem, where $d = |P + 1|$.
\end{proof}

We now define a certain class of primes.
Suppose $P$ is an ordered sequence of $i$ primes. Suppose $d$ is
an integer greater than $i$. Let $p$ be the largest prime such 
that there exists a $(P + p)$-initiated OPA having less than $d$ 
prime divisors. We know this prime $p$ exists since 
there exist only finitly many primitive abundant numbers having
fewer than $d$ prime divisors(Dickson). We denote such a prime $p$
as $Cap_d(P)$. 

\begin{theorem}
Let $P$ be an ordered sequence of primes, and  let $p$ and $q$ be 
primes such that $max(P) < r < s$. Suppose $d$ is an integer such
that $d > |P|$. Finally, suppose that $r > Cap_d(P)$. If there
does not exist any $(P + r)$-initiated OPN's with $d$ divisors,
then there does not exist any $(P + s)$-initiated OPN's with $d$ 
divisors.
\end{theorem}

\begin{proof}
Assume, to the contrary, that there does exist a $(P + 
s)$-initiated OPN with $d$ divisors. By Theorem {\ref{Divisibility}},
there exists a $(P + r)$-initated OPN having fewer than $d$
divisors. However, $r > Cap_d(P)$, and by definition of
$Cap_d(P)$, there cannot exist a $(P + r)$-initiated OPN having
fewer than $d$ divisors. This is a contradiction.
\end{proof}

\end{document}
